%préambule
\section{Méthode de Jacobi}
Rappelons que la méthode de \textbf{Jacobi} est itérative et ne garantit pas toujours un résultat. La méthode est définie si A est définie positive.\\
L'algorithme permet de trouver un résultat si la matrice est dite à diagonale strictement dominante. \\
Autrement dit, Soit $[a_{ij}]_{0 \leq i,j \leq n}$ les coefficients réels peuplant $A \in \mathcal{M}_{n,n} (\mathbb{R})$, alors si:\\
$\forall i, \vert a_{i,i} \vert <  \sum \limits_{i \neq j} \vert a_{ij} \vert$, on a que Jacobi converge vers l'unique solution du système $Ax=b$. 
\subsection{Principe de la méthode}
On veut résoudre $Ax=b$ avec $A \in \mathcal{M}_{n,n} (\mathbb{R})$, $n \in \mathbb{N}$, $x$ le vecteur colonne contenant les inconnus et $b$ le vecteur colonne des solution. \\
On pose $D \in \mathcal{M}_{n,n}(\mathbb{R})$ la matrice contenant les coefficients $[a_{i,j}]_{0 \leq i=j \leq n}$ de $A$. \\
On pose aussi $E$ et $F$ avec $E$ la matrice triangulaire opposée inférieure de $A$ et $F$ la matrice supérieure opposée de $A$.  \\
On obtient alors:
\begin{align}
Ax &=b \\
(D-E-F)x &= b \\
Dx -(E+F)x &= b \\
x &=D^{-1}(E+F)x+D^{-1}b \\
x^{k+1} &=D^{-1}(E+F)x^{k}+D^{-1}b 
\end{align}
Ce qui donne l'algorithme suivant: \\
Soit $\epsilon$ l'erreur maximale, un point initial $x^0$ et $k=0$ \\
avec $\epsilon^{0} = \vert \vert A x^0 -b \vert \vert$ \\
On obtient: 
\begin{lstlisting}[mathescape=true, frame=single]
Tant que ($\epsilon^{(k)} \leq \epsilon$)
        $x^{k+1}_i = \frac{1}{a_{ii}} [ b_i - \sum \limits_{j\neq i} a_{ij}x_j^{(k)}], i=1,\ldots,n$
        $\epsilon^{k+1} = \vert \vert Ax^{k+1} -b \vert \vert$
        $k=k+1$
FIN JACOBI
\end{lstlisting}
\textbf{Remarque, on ajoutera aussi un nombre d'itérations maximum afin de ne pas être dans le cas d'une boucle infinie (si Jacobi diverge alors l'erreur augmente)}.
\newpage
\subsection{Résolution manuelle}
\textit{Nous en détaillerons seulement une itération} \vspace{6pt}\\
Soit
$A= \begin{pmatrix}
4 & 1 & 0 \\
-1 & 3 & 6 \\
-2 & -5 & -3
\end{pmatrix}$,
$b = \begin{pmatrix}
8 \\
3 \\
8 \\
\end{pmatrix}$
,
et $x^{(0)} = \begin{pmatrix}
0 \\
0\\
0\\
\end{pmatrix}$ \vspace{8pt}\\
On a $A = \underbrace{\begin{pmatrix}
4 & 0 & 0 \\
0 & 3 & 0 \\
0 & 0 & -2
\end{pmatrix}}_{D}
- 
\underbrace{\begin{pmatrix}
0 & 0& 0 \\
1 & 0 & 0 \\
2 & 5 &0 
\end{pmatrix}}_E
- \underbrace{\begin{pmatrix}
0 & -1 & 0 \\
0 & 0 & -6\\
0 & 0 & 0
\end{pmatrix}}_F$ \vspace{8pt}\\
On a donc $x^{k+1}= D^{-1} [(E+F)x^{k} + b]$ \vspace{6pt}\\
Dans le cas présent on obtient alors: \vspace{6pt}\\
$x^1 = \begin{pmatrix}
\frac{1}{4} & 0 & 0 \\
0 & \frac{1}{3} & 0 \\
0 & 0 & -\frac{1}{2}
\end{pmatrix}
\left[ \left[ \begin{pmatrix}
0 & 0 & 0 \\
1 & 0  & 0 \\
2 & 5 & 0 \\
\end{pmatrix} + 
\begin{pmatrix}
0 & -1 & 0 \\
0 & 0 & -6 \\
0 & 0 & 0 
\end{pmatrix}
\right]
\begin{pmatrix}
0\\
0\\
0
\end{pmatrix}
+
\begin{pmatrix}
8 \\
3\\
8
\end{pmatrix}
  \right]$\vspace{8pt} \\
$x^1 = \begin{pmatrix}
2 \\
1 \\
-4
\end{pmatrix}$ \\
et
\begin{align*}
\epsilon^{(1)} & = \vert \vert Ax^{(1)} -b \vert \vert \\
\epsilon^{(1)} & = \vert \vert \begin{pmatrix}
4 & 1 & 0 \\
-1 & 3 & 6 \\
-2 & -5 & -3
\end{pmatrix}
\begin{pmatrix}
2\\
1\\
-4
\end{pmatrix} - 
\begin{pmatrix}
8 \\
3 \\
8
\end{pmatrix} \vert \vert \vspace{8pt}\\
\epsilon^{(1)} & = \vert \vert \begin{pmatrix}
1 \\
-26 \\
-5
\end{pmatrix} \vert \vert \vspace{8pt}\\
\epsilon^{(1)} & =  \sqrt{1^2+(-26)^2+(-5)^2} \\
\epsilon^{(1)} & =  \sqrt{(702)} \\
\end{align*}

\subsection{Implémentation}
Pour l'implémentation de cette méthode, nous utiliserons $\epsilon$ comme suit: \\
\begin{center}
$\epsilon^{(k)} = p^k = \text{Max}_{i=1,\ldots, n} \vert \bar{x_i} -\tilde{x_i}^k \vert $ \\
\end{center}
Où $\bar{x_i}$ sont les résultats attendus et $\tilde{x_i}^k$ est l'approximation trouvée à l'étape $k$. \\
De plus, on utilisera aussi une limite d'occurrence, pour pouvoir gérer les matrices où \textbf{Jacobi} diverge.\\
En l'occurrence, on se fixera un $\epsilon = 10^{-4}$ et un nombre d'itérations maximum fixé à $1000$.
\newpage
\subsection{Code}
\textit{On ne détaillera ici seulement les fonctions dites "non triviales" c'est-à-dire les fonctions étant en lien directe avec l'algorithme décrit.\\De plus chaque fonction présentée sera dépouillée de toute fonction d'affichage permettant de produire des chiffres liés à l'utilisation du programme (pour une question de lisibilité).}
\begin{lstlisting}[language=C,inputencoding=utf8, basicstyle=\fontsize{8}{10}\selectfont,caption=jacobi.c]
int jacobi(float **A, float *vector, float *b, float *S, int n, float minErr,int bound){
  float epsilon = epsi(S, vector, n);
  int k = 0;
  float *cp = malloc(n * sizeof *cp);
  while (k < bound && epsilon >= minErr) {
    fprintf(stderr, "%f ", epsilon);
    // printf("EPSILON : %f\n", epsilon);
    copy(cp, vector, n);
    for (int i = 0; i < n; i++) {
      vector[i] = (1 / A[i][i]) * (b[i] - jacobiSum(A, cp, n, i));
    }
    epsilon = epsi(vector, S, n);
    k++;
  }
  free(cp);
  return k;
}

\end{lstlisting}
\textbf{Commentaires:} Nous remarquerons que l'implémentation de l'algorithme de Jacobi est similaire à ce qui a été présenté précédemment. Pour des questions de lisibilité, $\epsilon^{(k)}$ est calculée par la fonction \textit{epsi()}, de même pour \textbf{jacobiSum()}, le calcul a été séparé afin de faciliter la compréhension du programme.
\begin{lstlisting}[language=C,inputencoding=utf8, basicstyle=\fontsize{8}{10}\selectfont,caption=jacobiSum() function in "source.h"]
float jacobiSum(float **A, float *V, int m, int i) {
  float s = 0;
  for (int j = 0; j < m; j++) {
    if (j != i)
      s += A[i][j] * V[j];
  }
  return s;
}
\end{lstlisting}
\textbf{Commentaire:} Permet de calculer de façon lisible est clair ceci: \\
\begin{center}
$\sum \limits_{j \neq i} a_{ij}x_j^{(k)}, i=1,\ldots k$
\end{center}
\newpage
\begin{lstlisting}[language=C,inputencoding=utf8, basicstyle=\fontsize{8}{10}\selectfont,caption=epsi() function in "source.h"]
float epsi(float *V, float *S, int n) {
  float max = Fabs(V[0] - S[0]);
  for (int i = 1; i < n; i++) {
    if (Fabs(S[i] - V[i]) > max)
      max = Fabs(S[i] - V[i]);
  }
  return max;
}
\end{lstlisting}
\textbf{Commentaires:} Utilise la fonction \textit{Fabs()} que nous avons implémenté, elle renvoie la valeur absolue d'un nombre flottant.\\
Cette fonction \textit{epsi()} permet donc de calculer l'erreur entre deux itérations de la façon suivante: \\
\begin{center}
$\epsilon^{(k)} = \text{Max}_{i=1,\ldots, n} \vert \bar{x_i} -\tilde{x_i}^k \vert $ \\
\end{center}
\begin{lstlisting}[language=C,inputencoding=utf8, basicstyle=\fontsize{8}{10}\selectfont,caption=conv() function in "source.h"]
int conv(float **A, int m) {
  float sl = 0;
  for (int i = 0; i < m; i++) {
    for (int j = 0; j < m; j++) {
      if (i != j)
        sl += fabs(A[i][j]);
    }
    if (A[i][i] - sl <= 0)
      return 0;
  }
  return 1;
}
\end{lstlisting}
\textbf{Commentaire} \textit{conv()} est une fonction utilitaire permettant d'effectuer une prédiction sur la convergeance potentielle d'une matrice par Jacobi.\\
Pour cela, on vérifie si la matrice sur laquelle on effectue Jacobi est à diagonale strictement dominante. \\
Ce qui se vérifie par cette formule: \\
\begin{center}
$\forall i, \vert a_{ii} \vert > \sum \limits_{j\neq i} \vert a_{ij} \vert$
\end{center}
\subsection{Entrées / Sorties}
\subsubsection{Entrées}
Le programme prend en entrée 2 paramètres à savoir: 
\begin{enumerate}
\item minErr, ou l'erreur minimale tolérée
\item bound, qui désigne la limite d'occurrences du programme (en cas de divergence)
\end{enumerate}
L'utilisateur peut ensuite décider de remplir manuellement sa matrice où de rediriger le flux d'un fichier comme suit:
\begin{lstlisting}[language=C,inputencoding=utf8, basicstyle=\fontsize{8}{10}\selectfont,caption=A4.txt]
3 3
7 6 9
4 5 -4
-7 -3 8 
\end{lstlisting}
Avec, sur la première ligne les dimensions de la matrice, suivi sur les lignes suivantes, des coefficients de la matrice.
\subsubsection{Sorties}
Le flux d'erreur sera réservé afin de produire des données engendrant des graphiques (des données formatées). Il s'agit de l'énumération de tous les $\epsilon$ calculés suivie du nombre d'itérations. \\
Sur les autres lignes seront inscrit des messages facilitant la prise en main du programme pour l'utilisateur. \\
Il figurera ensuite la prédiction quant à la convergence de la méthode.\\
Enfin, la dernière ligne représentera le nombre de $\epsilon$ calculés. \\
\subsection{Exécution}
Voici l'exécution sur les 12 matrices de la méthode de Jacobi avec $\epsilon=10^{-4}$.\\
\textit{La redirection du flux de sortie du programme est la suivante: >}
\begin{lstlisting}[language=C,inputencoding=utf8, basicstyle=\fontsize{8}{10}\selectfont,caption=Execution with A1 matrix]
 ===== CONVERGENCE PREDICTION: may not conv =====
PRINTING VECTOR FROM: 0x55ce917272e0 LOCATION :
-nan -48517597648316769037382673682594267136.000000 inf
EPSILON CALCULATED 742
\end{lstlisting}
\begin{lstlisting}[language=C,inputencoding=utf8, basicstyle=\fontsize{8}{10}\selectfont,caption=Execution with A2 matrix]
 ===== CONVERGENCE PREDICTION: may not conv =====
PRINTING VECTOR FROM: 0x55ea0288b2e0 LOCATION :
-nan -nan -inf
EPSILON CALCULATED 252
\end{lstlisting}
\begin{lstlisting}[language=C,inputencoding=utf8, basicstyle=\fontsize{8}{10}\selectfont,caption=Execution with A3 matrix]
 ===== CONVERGENCE PREDICTION: may not conv =====
PRINTING VECTOR FROM: 0x5579594bf2e0 LOCATION :
1.000052 1.000013 0.999954
EPSILON CALCULATED 13
\end{lstlisting}
\begin{lstlisting}[language=C,inputencoding=utf8, basicstyle=\fontsize{8}{10}\selectfont,caption=Execution with A4 matrix]
 ===== CONVERGENCE PREDICTION: may not conv =====
PRINTING VECTOR FROM: 0x5603748652e0 LOCATION :
1.000078 0.999903 1.000078
EPSILON CALCULATED 24
\end{lstlisting}
\begin{lstlisting}[language=C,inputencoding=utf8, basicstyle=\fontsize{8}{10}\selectfont,caption=Execution with A5 dimensions: 3x3]
 ===== CONVERGENCE PREDICTION: may not conv =====
PRINTING VECTOR FROM: 0x56131ea842e0 LOCATION :
1.000068 1.000045 1.000023
EPSILON CALCULATED 17
\end{lstlisting}
\begin{lstlisting}[language=C,inputencoding=utf8, basicstyle=\fontsize{8}{10}\selectfont,caption=Execution with A5 dimensions: 6x6]
 ===== CONVERGENCE PREDICTION: may not conv =====
PRINTING VECTOR FROM: 0x55b573d182e0 LOCATION :
0.999950 0.999927 0.999963 0.999982 0.999991 0.999995
EPSILON CALCULATED 18
\end{lstlisting}
\begin{lstlisting}[language=C,inputencoding=utf8, basicstyle=\fontsize{8}{10}\selectfont,caption=Execution with A5 dimensions: 8x8]
 ===== CONVERGENCE PREDICTION: may not conv =====
PRINTING VECTOR FROM: 0x56527ea952f0 LOCATION :
0.999949 0.999924 0.999962 0.999981 0.999991 0.999995 0.999998 0.999999
EPSILON CALCULATED 18
\end{lstlisting}
\begin{lstlisting}[language=C,inputencoding=utf8, basicstyle=\fontsize{6}{8}\selectfont,caption=Execution with A5 dimensions: 10x10]
 ===== CONVERGENCE PREDICTION: may not conv =====
PRINTING VECTOR FROM: 0x55cfc44702f0 LOCATION :
0.999949 0.999924 0.999962 0.999981 0.999990 0.999995 0.999998 0.999999 0.999999 1.000000
EPSILON CALCULATED 18
\end{lstlisting}
\begin{lstlisting}[language=C,inputencoding=utf8, basicstyle=\fontsize{8}{10}\selectfont,caption=Execution with A6 dimensions: 3x3]
 ===== CONVERGENCE PREDICTION: may not conv =====
PRINTING VECTOR FROM: 0x55774b4d72e0 LOCATION :
0.999956 0.999941 0.999911
EPSILON CALCULATED 24                                                                                              
\end{lstlisting}
\begin{lstlisting}[language=C,inputencoding=utf8, basicstyle=\fontsize{8}{10}\selectfont,caption=Execution with A6 dimensions: 6x6]
 ===== CONVERGENCE PREDICTION: may not conv =====
PRINTING VECTOR FROM: 0x562479ac82e0 LOCATION :
0.999989 0.999967 0.999949 0.999917 0.999917 0.999926
EPSILON CALCULATED 61
\end{lstlisting}
\begin{lstlisting}[language=C,inputencoding=utf8, basicstyle=\fontsize{8}{10}\selectfont,caption=Execution with A6 dimensions: 8x8]
 ===== CONVERGENCE PREDICTION: may not conv =====
PRINTING VECTOR FROM: 0x55a7d738a2f0 LOCATION :
0.999994 0.999985 0.999968 0.999954 0.999927 0.999918 0.999904 0.999936
EPSILON CALCULATED 84
\end{lstlisting}
\begin{lstlisting}[language=C,inputencoding=utf8, basicstyle=\fontsize{6}{8}\selectfont,caption=Execution with A6 dimensions: 10x10]
 ===== CONVERGENCE PREDICTION: may not conv =====
PRINTING VECTOR FROM: 0x55cc8327d2f0 LOCATION :
0.999997 0.999992 0.999983 0.999974 0.999955 0.999942 0.999918 0.999912 0.999902 0.999934
EPSILON CALCULATED 104
\end{lstlisting}
\subsection{Remarque sur les résultats}
On remarquera que lorsque Jacobi renvoie \textbf{NaN, inf ou des valeurs proches de la limite d'encodage du type float (8 octets)} alors il s'agit du cas où cette méthode diverge, donc aucun résultat fiable n'est envisageable.\\
On rappelera aussi que dans le cadre de ce programme, si la méthode renvoie des valeurs \textbf{extrêmement proches} de 1, alors le programme a trouvé une solution. 