\documentclass{report}
\setlength{\headheight}{24.1638pt}
%packages
\usepackage[french]{babel}
\usepackage[T1]{fontenc}
\usepackage[utf8]{inputenc}
\usepackage{mathtools}
\usepackage{amssymb}
\usepackage{hyperref}
\usepackage{float}
\usepackage{amsthm}
\usepackage{listings}
\usepackage{geometry}
\usepackage{setspace}
\usepackage{graphicx}
\usepackage{fancyhdr}
\usepackage{subcaption}
\usepackage{cleveref}
\usepackage{envmath}
\usepackage{mdframed}

%commands
\newtheorem{defi}{Définition}
\renewcommand{\thedefi}{\empty{}}

\renewcommand\headrulewidth{1pt}
\newcommand{\crule}[3][c]{%
    \par\noindent
    \makebox[\linewidth][#1]{\rule{#2\linewidth}{#3}}}
\renewcommand{\thechapter}{\Roman{chapter}}

%Style de page
\pagestyle{fancy}
\fancyhead[L]{}
\fancyhead[C]{}
\fancyhead[R]{\leftmark}
\allowdisplaybreaks
\geometry{hmargin=3cm,vmargin=2.5cm}


%préambule
\title{Application en Ingénierie et Programmation Numérique\\[0.2em]\textit{\Large{"Rendu II - Méthodes de résolution de systèmes linéaires"}}}
\author{VILLEDIEU Maxance et BESQUEUT Corentin}
\date{26 octobre 2023}
\begin{document}
\maketitle
\tableofcontents
%\listoffigures
\chapter{Résolution de systèmes linéaires par des Méthodes Directes: Méthode de Gauss}
Dans le cadre de ce premier TP, nous devions implémenter l'algorithme du \emph{Pivot de Gauss} en utilisant le langage de programmation C.
\section{Détail de l'algorithme}
Soient deux matrices $A \in \mathcal{M}_{m,m} \text{ et  } b \in \mathcal{M}_{m,1}$ . \\
L'algorithme de Gauss se décrit ainsi: \\
\begin{lstlisting}[mathescape=true, frame=single]
Pour $k = 1,\ldots,n-1$ Faire:
	Pour $i=k+1,\ldots,n$ Faire:

				$\alpha_i^{(k)} = \frac{a_{ik}^{(k)}}{a_{kk}^{(k)}}$ 
		
		Pour $j=k,\ldots,n$ Faire:

		   		$a_{ij}^{(k+1)} = a_{ij}^{(k)} - \alpha_i^{(k)}a_{kj}^{(k)}$
	
		FIN Pour $j$

				$b_i^{(k+1)} = b_i^{(k)} - \alpha_i^{(k)}b_k^{(k)}$

	FIN Pour $i$
FIN Pour $k$
\end{lstlisting}
\newpage
Une fois la matrice échelonnée par cet algorithme, on appliquera la formule suivante pour trouver les solutions du système : \\
\begin{mdframed}
\begin{center}
\begin{large}
$ x_n = \frac{b_n}{a_{m,m}}$ \\
\end{large}
\end{center}
et \\
\begin{center}
\begin{large}
$ \forall i = n-1, \ldots, 1$, $x_i = \frac{1}{a_{ii}}\left( b_i-\sum\limits_{j=1+i}^n a_{ij}x_j \right)$\\
\end{large}
\end{center}
\end{mdframed}
La complexité temporelle de cet algorithme est cubique soit $O(n^3)$ avec une complexité exacte de $\frac{2n^3}{3}$. \\
Pour l'implémentation de cet algorithme, nous allons présenter deux façons de le conceptualiser avec une comparaison algorithmique des deux programmes. \\
\section{Le pivot de Gauss en pratique}
\subsection{De manière générale}
Soit $A \in \mathcal{M}_{m,m}$ et $B \in \mathcal{M}_{m,1}$ et $x$ la matrice des inconnues.\\
Considérons alors le système suivant $Ax=b$.\\
Ce système peut être représenté sous la forme d'une matrice augmentée $M$ tel que: \vspace{10pt}\\
$
M =  \begin{pmatrix}
a_{11}& \ldots & a_{1m} &\bigm| & b_{1} \\
\vdots & \ddots & \vdots&\bigm| &  \vdots \\
a_{m1} & \ldots & a_{mm} &\bigm|& b_{m} 
\end{pmatrix}
$
\vspace{10pt}\\
Après exécution du pivot de Gauss, $M$ devient \vspace{10pt}\\
$
M =  \begin{pmatrix}
1& a_{12}& a_{13} &\ldots & a_{1m} &\bigm| & b_{1}^{'} \\
0 & 1 & a_{23}^{'}&  \ldots &a_{2m}^{'} &\bigm| & b_{2}^{'} \\
0 & 0 & 1 & \ldots &\vdots &\bigm| & b_{3}^{'} \\
\vdots & \vdots & \vdots& \ddots& \vdots &\bigm| &  \vdots \\
0 & \ldots &0 &\ldots &1&\bigm|& b_{m}^{'}
\end{pmatrix}
$  \vspace{10pt}\\
Une fois que tous les pivots sont placés, il suffira de reconstituer le système et de le remonter afin de déterminer les inconnues comme suit:\vspace{10pt}\\
Soit $A^{'}=  \begin{pmatrix}
1& a_{12}^{'}& a_{13}^{'} &\ldots & a_{1m}^{,} \\
0 & 1 & a_{23}^{'}&  \ldots &a_{2m}^{'} \\
0 & 0 & 1 & \ldots &\vdots\\
\vdots & \vdots & \vdots& \ddots& \vdots \\
0 & \ldots &0 &\ldots &1
\end{pmatrix}
$
et $b = \begin{pmatrix}
b_{1}^{'} \\
b_{2}^{'} \\
b_{3}^{'} \\
\vdots \\
 b_{m}^{'}
\end{pmatrix}
$ \\
alors pour $A^{'}x = b$ on a donc \begin{large}$ x_n = \frac{b_n}{a_{m,m}}$\end{large} et \begin{large}$x_i = \frac{1}{a_{ii}}\left( b_i-\sum\limits_{j=1+i}^n a_{ij}x_j \right), \forall i = n-1, \ldots, 1$\end{large}. \vspace{8pt}\\
\textit{Nous remarquerons que l'implémentation du pivot de Gauss ne nécessitera pas de mettre nos pivots à $1$}

\subsection{Exercice}
\begin{mdframed}
Résoudre le système linéaire suivant : 
\begin{System}
  x + y + 2z = 3 \\
  x + 2y + z = 1 \\
  2x + y + z = 0 
\end{System}
\end{mdframed}
On pose $A= 
\begin{pmatrix}
1 & 1 & 2 \\
1 & 2 & 1 \\
2 & 1 & 1 \\
\end{pmatrix}$, 
$B= 
\begin{pmatrix}
3 \\
1 \\
0 \\
\end{pmatrix}$ 
et
$X= 
\begin{pmatrix}
x \\
y \\
z \\
\end{pmatrix}$ \vspace{6pt}\\
puis la matrice augmentée 
$\begin{pmatrix}
    A & \big| & B \\
\end{pmatrix}
=
\begin{pmatrix}
    1 & 1 & 2 & \big| & 3 \\
    1 & 2 & 1 & \big| & 1 \\
    2 & 1 & 1 & \big| & 0 \\
\end{pmatrix}$  \vspace{8pt}\\
Commençons par échelonner la matrice augmentée à l'aide du pivot de Gauss: \vspace{8pt}\\
$
\begin{pmatrix}
    1 & 1 & 2 & \big| & 3 \\
    1 & 2 & 1 & \big| & 1 \\
    2 & 1 & 1 & \big| & 0 \\
\end{pmatrix}
\xrightarrow[L_2-L_1 \rightarrow L_2]{-(L_3+2L_1) \rightarrow L_3}
\begin{pmatrix}
    1 & 1 & 2 & \big| & 3 \\
    0 & 1 & -1 & \big| & -2 \\
    0 & 1 & 3 & \big| & 6 \\
\end{pmatrix}
\xrightarrow{\frac{L_3-L_2}{4} \rightarrow L_3}
\begin{pmatrix}
    1 & 1 & 2 & \big| & 3 \\
    0 & 1 & -1 & \big| & -2 \\
    0 & 0 & 1 & \big| & 2 \\
\end{pmatrix}
$ \vspace{8pt}\\
Maintenant que notre matrice augmentée est échelonnée, nous pouvons déterminer les inconnues du système par substitution (en partant du bas): \vspace{8pt}\\
$
\begin{pmatrix}
    1 & 1 & 2 & \big| & 3 \\
    0 & 1 & -1 & \big| & -2 \\
    0 & 0 & 1 & \big| & 2 \\
\end{pmatrix}
\xrightarrow[L_1-2L_3 \rightarrow L_1]{L_2+L_3 \rightarrow L_2}
\begin{pmatrix}
    1 & 1 & 0 & \big| & -1 \\
    0 & 1 & 0 & \big| & 0 \\
    0 & 0 & 1 & \big| & 2 \\
\end{pmatrix}
\xrightarrow{L_1-L_2 \rightarrow L_1}
\begin{pmatrix}
    1 & 0 & 0 & \big| & -1 \\
    0 & 1 & 0 & \big| & 0 \\
    0 & 0 & 1 & \big| & 2 \\
\end{pmatrix}
$ \vspace{8pt}\\

Nous avons maintenant $A=I_3$ et donc nous pouvons remplacer dans le système $AX=B$, $A$ par $I_3$, ce qui nous donne: \vspace{4pt}\\
$
AX=B \iff I_3X=B
\iff
    \begin{pmatrix}
        1 & 0 & 0\\
        0 & 1 & 0 \\
        0 & 0 & 1
    \end{pmatrix}
    \begin{pmatrix}
    	x \\
		y \\
		z \\
    \end{pmatrix}
    =
    \begin{pmatrix}
    	-1\\
		0 \\
		2 \\
    \end{pmatrix}
    \iff
    \begin{pmatrix}
    	x \\
		y \\
		z \\
    \end{pmatrix}
    =
    \begin{pmatrix}
    	-1\\
		0 \\
		2 \\
    \end{pmatrix}$
\vspace{6pt}\\
Nous avons alors notre couple solution du système, qui est : \vspace{4pt}\\
\begin{System}
  x = -1 \\
  y = 0 \\
  z = 2 
\end{System}\\

\newpage
\lstset{
  firstnumber=0, 
  numbers=left,               
  frame=single,
  language=C,                                       
  showstringspaces=false
}
\section{Implémentation de l'algorithme de Gauss en passant par le système d'équations linéaires}
\subsection{Code source}
Voici mon implémentation de l'algorithme de Gauss, qui n'utilise pas la matrice augmentée. En effet, l'algorithme travaille directement avec le système d'équations linéaires Ax=B.\\
\begin{lstlisting}[language=C,inputencoding=utf8, basicstyle=\fontsize{8}{10}\selectfont]
#include <stdio.h>
#include <string.h>
#include <stdlib.h>

/*
*CREATE A 2D FLOAT MATRIX
*/

float** createMatrix(int row, int column){
	float **mat=NULL;
	mat=malloc(row* sizeof(int*));
	if(mat==NULL){return NULL;}
	for (int i=0; i<row; i++){
		mat[i]=malloc(column* sizeof(int));
		if(mat[i]==NULL){
			for(int j=0; j<i; j++){
				free(mat[j]);
				return NULL;
			}
		}
	}
	return mat;
}

/*
*PRINT A 2D FLOAT MATRIX
*/

void printMatrix(float **mat, int row, int column){
	for (int i=0; i<row; i++){
    		for(int j=0; j<column; j++){
         		printf("%f   ", mat[i][j]);
    		}
    		printf("\n");
	}
}

/*
*FREE A 2D FLOAT MATRIX
*/

void freeMatrix(float **mat, int row){
	for(int i=0; i<row;i++){
		free(mat[i]);
	}
	free(mat);
}

/*
*COMPLETE A 2D FLOAT MATRIX FROM USER INPUT
*/

void completeMatrix(float **mat, int row, int column){
	for (int i=0; i<row; i++){
    		for(int j=0; j<column; j++){
    			printf("Coefficient at M_%d,%d:   ", i+1, j+1);
         		scanf("%f", &mat[i][j]);
    		}
	}
}

/*
*GENERATE A COLUMN VECTOR "B" FROM A 2D FLOAT MATRIX "A"
*/

void generateB(float **matA, float **matB, int row, int column){
	for(int i=0; i<row; i++){
		float sum=0;
		for(int j=0; j<column; j++){
			sum+=matA[i][j];
		}
		matB[i][0]=sum;
    	}
}

/*
*PERFORM GAUSSIAN ELIMINATION ON A Ax=B MATRIX SYSTEM OF LINEAR EQUATIONS
*/

void gauss(float** matA, float** matb, int size){
	for(int k=0; k<size-1; k++){
		for(int i=k+1; i<size; i++){
			float alpha=matA[i][k]/matA[k][k];
			for(int j=k; j<size; j++){
				matA[i][j]=matA[i][j]-alpha*matA[k][j];
			}
			matb[i][0]=matb[i][0]-alpha*matb[k][0];
		}
	}
}

/*
*SOLVE A MATRIX SYSTEM OF LINEAR EQUATIONS USING BACKWARD SUBSTITUTION
*/
void resolution(float** matA, float** matb, float** matx, int size){
	matx[size-1][0]=matb[size-1][0]/matA[size-1][size-1];
	for (int i=size-2; i>=0; i--){
		float sum=0;
		for(int j=i+1; j<size; j++){
			sum+=matA[i][j]*matx[j][0];
		}
		matx[i][0]=(1/matA[i][i])*(matb[i][0]-sum);
	}
}

int main(){

	//A Matrix
	int rowA;
	int columnA;
	printf("\nRow count of matrix A : ");
	scanf("%d", &rowA);
	printf("\nColumn count of matrix A : ");
	scanf("%d", &columnA);

	float** Amatrix=createMatrix(rowA, columnA);
	completeMatrix(Amatrix, rowA, columnA);
	puts("\n		A matrix \n");
	printMatrix(Amatrix, rowA, columnA);
	
	//B Matrix
	float** Bmatrix=createMatrix(rowA, 1);
	generateB(Amatrix, Bmatrix, rowA, columnA);
	puts("\n		B matrix \n");
	printMatrix(Bmatrix, rowA, 1);
	
	//X Matrix
	float** Xmatrix=createMatrix(rowA, 1);
	
	//Matrix Triangularization
	puts("\n		TRIANGULARIZATION \n");
	gauss(Amatrix, Bmatrix, rowA);
	puts("\n		A Matrix \n");
	printMatrix(Amatrix, rowA, columnA);
	puts("\n		B Matrix \n");
	printMatrix(Bmatrix, rowA, 1);
	
	//Solve the system
	puts("\n		SOLVING \n");
	resolution(Amatrix, Bmatrix, Xmatrix, rowA);
	puts("\n		SOLUTION VECTOR X \n");
	printMatrix(Xmatrix, rowA, 1);
	
	//Free
	freeMatrix(Amatrix, rowA);
	freeMatrix(Bmatrix, rowA);
	freeMatrix(Xmatrix, rowA);
	return 0;
}
\end{lstlisting}
\subsection{Commentaires}
\subsubsection{Fonctions usuelles de manipulation de matrices}\label{fonctusu}
Ce code implémente diverses fonctions pour travailler avec des matrices à coefficients en nombre flottants.\\
\begin{itemize}
\item La fonction \textit{\textbf{createMatrix}} alloue dynamiquement de la mémoire pour créer une matrice de nombres flottants avec un nombre spécifié de lignes et de colonnes.
\item La fonction \textit{\textbf{printMatrix}} affiche les éléments d'une matrice de nombres flottants.
\item La fonction \textit{\textbf{freeMatrix}} libère la mémoire allouée pour une matrice de nombres flottants.
\item La fonction \textit{\textbf{completeMatrix}} permet à l'utilisateur de saisir des valeurs pour remplir les éléments d'une matrice de nombres flottants.
\item La fonction \textit{\textbf{generateB}} génère un vecteur colonne $B$ en fonction de la somme des éléments de chaque ligne de la matrice $A$.
\end{itemize}

\subsubsection{Fonctions résolvant notre système linéaire $Ax=B$ à l'aide de l'algorithme de Gauss}
Dans le cadre de notre résolution de systèmes d'équations linéaires, deux fonctions jouent des rôles clefs dans ce code : la fonction \textit{\textbf{gauss}} et la fonction \textit{\textbf{resolution}}.\\

\begin{itemize}
\item La fonction \textit{\textbf{gauss}} joue un rôle important dans la préparation de la résolution de notre système d'équations linéaires. En effectuant l'élimination de Gauss sur la matrice $A$, elle la transforme en une matrice triangulaire supérieure. Cela signifie que les éléments sous la diagonale principale de la matrice deviennent tous des zéros, simplifiant ainsi la résolution du système. De plus, la fonction met également à jour la matrice $B$ en conséquence, garantissant que notre système $Ax=B$ reste équilibré.\\

\item La fonction \textit{\textbf{resolution}}, quant à elle, prend en charge la résolution effective du système linéaire une fois que la matrice $A$ a été triangulée par la fonction \textbf{gauss}. Elle utilise la méthode de substitution pour calculer la solution et stocke le résultat dans le vecteur $X$. Cette étape finale permet d'obtenir les valeurs des variables inconnues du système, fournissant ainsi la solution recherchée pour le problème initial.\\
\end{itemize}


En combinant ces deux fonctions avec celles citées dans la sous-section \ref{fonctusu}, le code réalise un processus complet de résolution de systèmes d'équations linéaires de manière efficace et précise (aux erreurs d'arrondies près).\\
\subsection{Interaction Utilisateur/Console}
\subsubsection{Entrées utilisateur}

En premier lieu dans notre programme, nous avons besoin de spécifier le système $Ax=B$ à l'ordinateur. 
Pour ce faire, nous allons dans l'ordre :
\begin{enumerate}
\item Allouer une matrice $A$ en mémoire. Cette matrice verra sa taille définie par la première entrée utilisateur du programme (nous demanderons consécutivement le nombre de lignes, puis le nombre de colonnes de la matrice).
\item Définir les coefficients de la matrice $A$. Il s'agira de la deuxième entrée utilisateur de notre programme. \\
Par définition de notre fonction \textit{\textbf{completeMatrix}}, nous remplirons la matrice dans l'ordre suivant:\\
$a_{1,1}, a_{1,2}, ..., a_{1,n}, \text{   puis   } a_{2,1}, ... a_{2,n}, \text{   jusque   }  a_{n,1}, ..., a_{n,n}$
\item Allouer une matrice $B$ en mémoire. À noter que la taille de $B$ est définie automatiquement en fonction de la taille de $A$. Nous avons $A\in \mathcal{M}_{n,p} \Rightarrow B\in \mathcal{M}_{n,1}$.
\item Définir les coefficients de la matrice $B$. Chaque coefficient prendra la valeur de la somme des éléments de la ligne respective de la matrice $A$.\\
Nous avons donc:\\ $ \text{Soient } A\in \mathcal{M}_{n,p} \text{ et } B\in \mathcal{M}_{n,1}  , \forall i \in \{1,n\}  , b_{i,1}=\sum_{j=1}^{p} a_{i,p}$.\\
\item Allouer une matrice $X$ en mémoire. Cette matrice aura la même taille que la matrice $B$. Ces coefficients ne seront pas définis pour le moment.
\end{enumerate}

En guise d'exemple, le système matriciel $AX=B$ suivant:

\begin{equation}
\begin{pmatrix}
3 & 0 & 4\\
7 & 4 & 2 \\
-1 & 1 & 2
\end{pmatrix} 
\begin{pmatrix}
x_1\\
x_2\\
x_3\\
\end{pmatrix}
=
\begin{pmatrix}
7 \\
13 \\
2
\end{pmatrix}
\end{equation}
\\


est représenté par l'entrée utilisateur:
\begin{lstlisting}[caption=User Input, basicstyle=\fontsize{6}{8}\selectfont]
Row count of matrix A : 3

Column count of matrix A : 3

		FILL IN THE VALUE OF MATRIX A 

Value for a_1,1:   3
Value for a_1,2:   0
Value for a_1,3:   4
Value for a_2,1:   7
Value for a_2,2:   4
Value for a_2,3:   2
Value for a_3,1:   -1
Value for a_3,2:   1
Value for a_3,3:   2

\end{lstlisting}
Une fois toutes les matrices initialisées et complétées, nous pouvons attaquer la résolution du système par la triangularisation du système. Ceci fait, nous résolverons le système obtenu pour obtenir notre vecteur $X$ solution.\\
\subsubsection{Affichage Console}
Dès lors le système $AX=B$ connu par l'ordinateur, ce dernier peut retrouver les valeurs de la matrice $X$. Voici l'affichage produit par notre programme en console: \\
\begin{lstlisting}[caption=Console Display of the Gauss elimination for the AX=B system mentioned above, basicstyle=\fontsize{6}{8}\selectfont]
		A matrix 

3.000000   0.000000   4.000000   
7.000000   4.000000   2.000000   
-1.000000   1.000000   2.000000   

		B matrix 

7.000000   
13.000000   
2.000000   

		TRIANGULARIZATION 
		A Matrix 

3.000000   0.000000   4.000000   
0.000000   4.000000   -7.333333   
0.000000   0.000000   5.166667   

		B Matrix 

7.000000   
-3.333332   
5.166667   

		SOLVING 
		SOLUTION VECTOR X 

1.000000   
1.000000   
1.000000   

\end{lstlisting}
Il est à repérer que le programme affiche dans cet ordre: \\
\begin{itemize}
\item La Matrice $A$ 
\item La Matrice $B$
\item La Matrice $A$ une fois triangulée supérieure
\item La Matrice $B$ une fois mise à jour en conséquence pour que le système reste équilibré
\item La Matrice $X$ solution du système
\end{itemize}

\textit{\underline{Remarque}: le temps d'exécution de ce programme a été de 0.000237 secondes}
\subsection{Exemples d'exécutions}

Soient les matrices suivantes données dans le TP:\\

$A_2 = \begin{pmatrix}
-3 & 3 & -6 \\
-4 & 7 &  8 \\
5 & 7 & -9 \\
\end{pmatrix}
$
,
$A_4 = \begin{pmatrix}
7 & 6 & 9 \\
4 & 5 &  -4\\
-7 & -3 & 8 \\
\end{pmatrix}
$,
$A_6 = \begin{pmatrix}
-3 & 3 & -6 \\
-4 & 7 &  8 \\
5 & 7 & -9 \\
\end{pmatrix}
$
\vspace{12pt}\\
On obtient respectivement les résultats suivants:
\\
\begin{lstlisting}[caption={$A_2X=B$} results, basicstyle=\fontsize{4}{6}\selectfont]
  		A matrix 

-3.000000   3.000000   -6.000000   
-4.000000   7.000000   8.000000   
5.000000   7.000000   -9.000000   

		B matrix 

-6.000000   
11.000000   
3.000000   

		TRIANGULARIZATION
		A Matrix 

-3.000000   3.000000   -6.000000   
0.000000   3.000000   16.000000   
0.000000   0.000000   -83.000000   

		B Matrix 

-6.000000   
19.000000   
-83.000000   

		SOLVING 
		SOLUTION VECTOR X 

1.000000   
1.000000   
1.000000   

Temps d'execution : 0.000250 secondes
\end{lstlisting}
\begin{lstlisting}[caption={$A_4X=B$} results, basicstyle=\fontsize{4}{6}\selectfont]
		A matrix 

7.000000   6.000000   9.000000   
4.000000   5.000000   -4.000000   
-7.000000   -3.000000   8.000000   

		B matrix 

22.000000   
5.000000   
-2.000000   

		TRIANGULARIZATION 
		A Matrix 

7.000000   6.000000   9.000000   
0.000000   1.571428   -9.142858   
0.000000   0.000000   34.454552   

		B Matrix 

22.000000   
-7.571429   
34.454548   

		SOLVING 
		SOLUTION VECTOR X 

1.000001   
0.999999   
1.000000   

Temps d'execution : 0.000231 secondes
\end{lstlisting}
\begin{lstlisting}[caption={$A_6X=B$} results, basicstyle=\fontsize{4}{6}\selectfont]
		A matrix 

-3.000000   3.000000   -6.000000   
-4.000000   7.000000   8.000000   
5.000000   7.000000   -9.000000   

		B matrix 

-6.000000   
11.000000   
3.000000   

		TRIANGULARIZATION 
		A Matrix 

-3.000000   3.000000   -6.000000   
0.000000   3.000000   16.000000   
0.000000   0.000000   -83.000000   

		B Matrix 

-6.000000   
19.000000   
-83.000000   

		SOLVING 
		SOLUTION VECTOR X 

1.000000   
1.000000   
1.000000   

Temps d'execution : 0.000246 secondes
\end{lstlisting}

\textbf{On remarquera} que sur le calcul de $A_4$, on tombe sur des valeur extrêmement proche de $1$. Ceci est provoqué à cause des erreurs d'arrondis provoqués par l'encodage des nombres flottants.     
\newpage
\documentclass{report}
\usepackage[T1]{fontenc}
\usepackage[utf8]{inputenc}
\usepackage{mathtools}
\usepackage{amssymb}
\usepackage{hyperref}
\usepackage{float}
\usepackage{amsthm}
\usepackage{listings}
\usepackage{geometry}
\usepackage{setspace}
\usepackage{graphicx}
\usepackage{fancyhdr}
\usepackage{subcaption}
\usepackage{cleveref}

\begin{document}
\subsection{Implémentation grâce à une matrice augmentée}
\subsubsection{Code source}
Voici le code source de mon implémentation du pivot de Gauss via le passage par la matrice augmentée.\\
C'est-à-dire que dans mon implémentation il y a une concaténation des matrices.
\begin{lstlisting}[language=C]
#include <stdio.h>
#include <stdlib.h>
#include <string.h>
#include <time.h>
/*
 * PRINT MATRIX WITH RIGHT FORMAT
 */
void printMatrix(float **matrix, int m, int p) {
  printf("PRINTING MATRIX FROM: %p LOCATION :\n", matrix);
  for (int i = 0; i < m; i++) {
    for (int j = 0; j < p; j++) {
      (j <= p - 2) ? printf("%f ", matrix[i][j]) : printf("%f", matrix[i][j]);
    }
    puts("");
  }
}
/*
 * ALLOCATE MEMORY FOR MATRIX
 */
float **allocate(int m, int n) {
  float **T = malloc(m * sizeof *T);
  for (int i = 0; i < m; i++) {
    T[i] = malloc(n * sizeof *T[i]);
    if (T[i] == NULL) {
      for (int j = 0; j < i; j++) {
        free(T[i]);
      }
      free(T);
      puts("ALLOCATION ERROR");
      exit(-1);
    }
  }
  return T;
}
/*
 * FILL MATRIX BY USER INPUT
 */
void fillM(int m, int p, float **T) {
  for (int i = 0; i < m; i++) {
    for (int j = 0; j < p; j++) {
      T[i][j] = 0;
      printf("Enter coefficient for %p[%d][%d]", T, i, j);
      scanf("%f", &T[i][j]);
    }
  }
}
/*
 * FREE MATRIX
 */
void freeAll(float **T, int m) {
  for (int i = 0; i < m; i++) {
    free(T[i]);
  }
  free(T);
}
/*
 * IMPLEMENTATION OF '.' OPERATOR FOR MATRIX
 */
float **multiplication(float **M1, float **M2, int m, int q) {
  float **R = allocate(m, q);
  for (int i = 0; i < m; i++) {
    for (int j = 0; j < q; j++) {
      for (int k = 0; k < q; k++) {
        R[i][j] += M1[i][k] * M2[k][j];
      }
    }
  }
  return R;
}
/*
 * BUILD AUGMENTED MATRIX
 */
float **AugmentedMatrix(float **M1, float **M2, int m, int n) {
  float **A = allocate(m, m + 1);
  for (int i = 0; i < m; i++) {
    for (int j = 0; j < n + 1; j++) {
      (j != n) ? (A[i][j] = M1[i][j]) : (A[i][j] = M2[i][0]);
    }
  }
  return A;
}
/*
 * PERFORM GAUSS ALGORITHM ONLY ON AUGMENTED MATRIX
 */
void gauss(float **A, int m, int p) {
  if (m != p) {
    puts("La matrice doit etre carree !");
    return;
  }
  for (int k = 0; k <= m - 1; k++) {
    for (int i = k + 1; i < m; i++) {
      float pivot = A[i][k] / A[k][k];
      for (int j = k; j <= m; j++) {
        A[i][j] = A[i][j] - pivot * A[k][j];
      }
    }
  }
}
/*
 * DETERMINE ALL UNKNOWNS VARIABLES
 */
float *findSolutions(float **A, int m) {
  float *S = calloc(m, sizeof *S);
  S[m - 1] = A[m - 1][m] / A[m - 1][m - 1];
  for (int i = m - 1; i >= 0; i--) {
    S[i] = A[i][m];
    for (int j = i + 1; j < m; j++) {
      S[i] -= A[i][j] * S[j];
    }
    S[i] = S[i] / A[i][i];
  }
  return S;
}
int main() {
  int m, n, p, q;
  float **P, **Q, **B, **A, *S;
  clock_t start, end;
  double execution;
  puts("Nombre de ligne suivit du nombre de colonne pour la matrice 1:");
  scanf("%d%d", &m, &p);
  puts("Nombre de ligne suivit du nombre de colonne pour la matrice 2:");
  scanf("%d%d", &n, &q);
  start = clock();
  P = allocate(m, p);
  Q = allocate(n, q);
  fillM(m, p, P);
  fillM(n, q, Q);
  printMatrix(P, m, p);
  printMatrix(Q, n, q);
  B = multiplication(P, Q, m, n);
  printMatrix(B, m, q);
  A = AugmentedMatrix(P, B, m, p);
  gauss(A, m, p);
  printMatrix(A, m, m + 1);
  S = findSolutions(A, m);
  puts("SOLUTIONS");
  for (int i = 0; i < m; i++)
    printf("x%d = %f\n", i, S[i]);
  freeAll(P, m);
  freeAll(Q, n);
  freeAll(B, m);
  freeAll(A, m);
  free(S);
  end = clock();
  execution = ((double)(end - start) / CLOCKS_PER_SEC);
  printf("RUNTIME: %f seconds", execution / 10);
  return 0;
}

\end{lstlisting}
\subsubsection{Commentaires du code}
Mon implémentation utilise strictement l'algorithme de Gauss rappelé précédemment avec seulement quelques changements d'indices puisque au lieu de travailler sur une matrice carrée et un vecteur colonne, mon programme utilise une matrice augmentée ayant $m$ lignes et $m+1$ colonnes, $m\in \mathbb{N}^*$. \\
\textbf{Détail des fonctions non conventionnelles:}\\
\textit{Comme mentioné précédemment, je ne détaillerai pas les fonction gauss() et findSolutions() puisque ces fonctions permettent strictement que d'une part d'implémenter l'algorithme de Gauss et d'autre part à "remonter" la matrice échelonnée afin de récupérer les valeurs des inconnus}. \\
-\textbf{float **AugmentedMatrix(float **M1, float **M2, int m, int n):} cette fonction permet de créer une matrice $A \in \mathcal{M}_{m,m+1}$ à partir de la concaténation de $M1 \in \mathcal{M}_{mm}$ et $M2 \in \mathcal{M}_{m,1}$. \\
Soient $a_{ij}$ les coefficients peuplant $A$, $b_{ij}$ les coefficients peuplant $M1$ et $c_{i0}$ les coefficients peuplant $M2$. \\
On obtient alors $a_{ij} = b_{ij} \forall i \in \mathbb{N}_{m}, \forall j \in \mathbb{N}_{m}$ et $a_{ij} = c_{i0}$ si $j = m+1$. \\
Cette fonction renvoie alors $A$, la matrice de floattant créee dynamiquement.
\subsubsection{Inputs / Outputs}
Mon programme demande d'abord 4 entiers $m,p,n,q$ entiers qui correspondent aux dimensions de la première matrice $A \in \mathcal{M}_{mp}$ et de la seconde matrice $X \in \mathcal{M}_{nq}$. Le but étant de résoudre le système $AX=b$, nous initialiserons $X$ à $1$. Ce choix de valeur permettra de controller la validité du programme, ainsi si à la fin du programme $\forall x_i \neq 1, \forall i \in \mathbb{N}_{n}$, on pourra affirmer que le programme est faux.  \\
Sur le $m\times p$ prochaines lignes, le programme demande les coefficients de $A$. \\
Sur les $n \times q$ prochaines lignes, le programme demandera les coefficient du vecteur colonne $X$, que l'utilisateur initialisera à $1$. \\
On peut alors automatiser les entrée en utilisant des fichiers. \\
Ainsi la matrice:
$ M = \begin{pmatrix}
3 & 0 & 4\\
7 & 4 & 2 \\
-1 & 1 & 2
\end{pmatrix}
$
et 
$ X = \begin{pmatrix}
1 \\
1 \\
1 \\
\end{pmatrix}
$ \\
sont représentés par ce fichier d'entrée:
\newpage
\begin{lstlisting}[caption=input.txt]
3 3
3 1 
3 0 4
7 4 2
-1 1 2
1 1 1
\end{lstlisting}
Pour ce qui est des résultats produits par mon programme, une fois injecté dans un fichier texte, une input "type" ressemble à ceci. \\
\begin{lstlisting}[caption=Gauss elimination with M and X matrix]
PRINTING MATRIX FROM: 0x556d938672c0 LOCATION :
3.000000 0.000000 4.000000
7.000000 4.000000 2.000000
-1.000000 1.000000 2.000000
PRINTING MATRIX FROM: 0x556d93867340 LOCATION :
1.000000
1.000000
1.000000
PRINTING MATRIX FROM: 0x556d938673c0 LOCATION :
7.000000
13.000000
2.000000
PRINTING MATRIX FROM: 0x556d93867440 LOCATION :
3.000000 0.000000 4.000000 7.000000
0.000000 4.000000 -7.333333 -3.333332
0.000000 0.000000 5.166667 5.166667
SOLUTIONS
x0 = 1.000000
x1 = 1.000000
x2 = 1.000000
RUNTIME: 0.000002 seconds 
\end{lstlisting}
On remarquera que le programme affiche dans cet ordre: \\
- \textbf{La Matrice A} \\
- \textbf{La Matrice X} \\
- \textbf{La Matrice B trouvé avec les valeur de X} \\
- \textbf{La Matrice augmentée en triangle supérieur} \\
- \textbf{Les solutions} \\
- \textbf{Un timer permettant de contrôler le temps d'exécution approximatif de mon programme} 
\subsection{Exemples d'exécutions}

Soient $A_2 = \begin{pmatrix}
-3 & 3 & -6 \\
-4 & 7 &  8 \\
5 & 7 & -9 \\
\end{pmatrix}
$
,
$A_4 = \begin{pmatrix}
7 & 6 & 9 \\
4 & 5 &  -4\\
-7 & -3 & 8 \\
\end{pmatrix}
$,
$A_6 = \begin{pmatrix}
-3 & 3 & -6 \\
-4 & 7 &  8 \\
5 & 7 & -9 \\
\end{pmatrix}
$
\\
On obtient respectivement ces résulats:
\\
\begin{lstlisting}[caption=Matrix 2 results]
PRINTING MATRIX FROM: 0x55f604fb32c0 LOCATION :
-3.000000 3.000000 -6.000000
-4.000000 7.000000 8.000000
5.000000 7.000000 -9.000000
PRINTING MATRIX FROM: 0x55f604fb3340 LOCATION :
1.000000
1.000000
1.000000
PRINTING MATRIX FROM: 0x55f604fb33c0 LOCATION :
-6.000000
11.000000
3.000000
PRINTING MATRIX FROM: 0x55f604fb3440 LOCATION :
-3.000000 3.000000 -6.000000 -6.000000
0.000000 3.000000 16.000000 19.000000
0.000000 0.000000 -83.000000 -83.000000
SOLUTIONS
x0 = 1.000000
x1 = 1.000000
x2 = 1.000000
RUNTIME: 0.000002 seconds   
\end{lstlisting}
\begin{lstlisting}[caption=Matrix 4 results]
PRINTING MATRIX FROM: 0x55f7afd662c0 LOCATION :
7.000000 6.000000 9.000000
4.000000 5.000000 -4.000000
-7.000000 -3.000000 8.000000
PRINTING MATRIX FROM: 0x55f7afd66340 LOCATION :
1.000000
1.000000
1.000000
PRINTING MATRIX FROM: 0x55f7afd663c0 LOCATION :
22.000000
5.000000
-2.000000
PRINTING MATRIX FROM: 0x55f7afd66440 LOCATION :
7.000000 6.000000 9.000000 22.000000
0.000000 1.571428 -9.142858 -7.571429
0.000000 0.000000 34.454552 34.454548
SOLUTIONS
x0 = 1.000001
x1 = 0.999999
x2 = 1.000000
RUNTIME: 0.000002 seconds  
\end{lstlisting}
\begin{lstlisting}[caption=Matrix 6 results]
PRINTING MATRIX FROM: 0x557fdaa552c0 LOCATION :
3.000000 -1.000000 0.000000
0.000000 3.000000 -1.000000
0.000000 -2.000000 3.000000
PRINTING MATRIX FROM: 0x557fdaa55340 LOCATION :
1.000000
1.000000
1.000000
PRINTING MATRIX FROM: 0x557fdaa553c0 LOCATION :
2.000000
2.000000
1.000000
PRINTING MATRIX FROM: 0x557fdaa55440 LOCATION :
3.000000 -1.000000 0.000000 2.000000
0.000000 3.000000 -1.000000 2.000000
0.000000 0.000000 2.333333 2.333333
SOLUTIONS
x0 = 1.000000
x1 = 1.000000
x2 = 1.000000
RUNTIME: 0.000002 seconds                       
\end{lstlisting}
\textbf{On remarquera} que sur le calcul de $A_4$, on tombe sur des valeur extrêmement proche de $1$. Ceci est provoqué à cause des erreurs d'arrondis provoqués par l'encodage des nombres flottants.
\end{document}      
\\
Nous remarquerons que l'implémentation utilisant la matrice augmentée est sensiblement meilleure en terme d'efficacité. En effet, le temps d'exécution est multiplié par $100$ sur la première implémentation.
\chapter{Résolution de systèmes linéaires par des Méthodes Itératives}
\documentclass{report}
\setlength{\headheight}{24.1638pt}
%packages
\usepackage[french]{babel}
\usepackage[T1]{fontenc}
\usepackage[utf8]{inputenc}
\usepackage{mathtools}
\usepackage{amssymb}
\usepackage{hyperref}
\usepackage{float}
\usepackage{amsthm}
\usepackage{listings}
\usepackage{geometry}
\usepackage{setspace}
\usepackage{graphicx}
\usepackage{fancyhdr}
\usepackage{subcaption}
\usepackage{cleveref}

%commands
\newtheorem{defi}{Définition}
\renewcommand{\thedefi}{\empty{}}

\renewcommand\headrulewidth{1pt}
\newcommand{\crule}[3][c]{%
    \par\noindent
    \makebox[\linewidth][#1]{\rule{#2\linewidth}{#3}}}
\renewcommand{\thechapter}{\Roman{chapter}}

%Style de page
\pagestyle{fancy}
\fancyhead[L]{}
\fancyhead[C]{}
\fancyhead[R]{\leftmark}
\allowdisplaybreaks
\geometry{hmargin=3cm,vmargin=2.5cm}


%préambule
\begin{document}

\lstset{
  firstnumber=0, 
  numbers=left,               
  frame=single,
  language=C,                                       
  showstringspaces=false
}
\section{Méthode de Gauss-Seidel}
\subsection{Introduction à la méthode de Gauss-Seidel}
La méthode de Gauss-Seidel est une méthode itérative pour résoudre 
les systèmes linéaires de la forme $Ax=b$, où $A$ est une matrice carrée d'ordre $n$ et $x, b$ sont des vecteurs de $\mathbb{R}^n$. 
C'est une méthode qui génère 
une suite qui converge vers la solution de ce système lorsque celle-ci en a une et lorsque les conditions de convergence suivantes sont satisfaites (quels que soient le vecteur $b$ et le point initial $x^0$):
\begin{itemize}
  \item Si la matrice $A$ est symétrique définie positive,
  \item Si la matrice $A$ est à diagonale strictement dominante.
\end{itemize}
\subsection{Mise en place des matrices pour la méthode de Gauss-Seidel}\label{decompMatrice}
Soit $Ax=b$ le système linéaire à résoudre, où $A\in \mathcal{M}_{n,n}$ et $b\in  \mathcal{M}_{n,1}$. On cherche $x\in \mathcal{M}_{n,1}$ solution du système.
Dans un premier temps, on va écrire $A$ sous la forme $A=D-E-F$ où $D$ est une matrice diagonale, $E$ est une matrice triangulaire inférieure, et $F$ est une matrice triangulaire supérieure. \\
On peut alors écrire:
\begin{align}
  Ax&=b \\
  \Leftrightarrow  (D-E-F)x&=b \\
  \Leftrightarrow  Dx&=b-(E+F)x \\
  \Leftrightarrow  x&=D^{-1}[b-(E+F)x]
\end{align}
On définit ensuite une suite de vecteurs $(x^k)$ en choisissant un vecteur $x^0$ et par la formule de récurrence:\\
\begin{equation}
  x_i^{k+1}=\frac{1}{a_{i,i}}\Bigg(b_i-\sum \limits_{j = 1}^{i-1}a_{i,j}x_{j}^{k+1} - \sum \limits_{j = i+1}^{n}a_{i,j}x_{j}^{k}\Bigg)
\end{equation}
\subsection{Algorithme}
Pour résoudre un système $Ax=b$, avec $A \in \mathcal{M}_{n}$ et $b\in \mathcal{M}_{n,1}$, on s'appuie sur l'algorithme suivant en posant :
\begin{itemize}
  \item un vecteur initial $x^{(0)}$ choisi au préalable,
  \item l'erreur à l'itération k=0 calculée par $\varepsilon^{(0)}=\Vert Ax^{(0)}-b\Vert$,
  \item une variable $k$ qui sera notre compteur d'itération.
\end{itemize}\vspace{6pt}
\label{algogs}
\begin{lstlisting}[mathescape=true, frame=single, basicstyle=\linespread{1.5}\fontsize{8}{10}\selectfont]
$x^{(0)}=x_0 \in \mathcal{M}_{n,1}$
$\varepsilon^{(0)}=\varepsilon \text{ (erreur)}$
$k=0$
Tant Que $(\varepsilon^{(k)} >= \varepsilon)$ faire:
      Pour $i=1$ a $n$:
            $x_{i}^{(k+1)}=\frac{1}{a_{i,i}}\Bigg[b_i-\Bigg(\sum \limits_{j=i+1}^{n}a_{i,j}x_j^{(k)} + \sum \limits_{j=1}^{i-1}a_{i,j}x_j^{(k+1)}\Bigg)\Bigg]   \text{  pour  } i=1, ..., n$
      $\varepsilon^{(k+1)}=\Vert Ax^{(k+1)}-b\Vert$
      $k=k+1$
Fin Tant Que
\end{lstlisting}
\subsection{Résolution manuelle}
\vspace{10pt}
Soit $A=
\begin{pmatrix}
  1 & 2 & 2 \\
  1 & 3 & 3 \\
  3 & 7 & 8 \\
\end{pmatrix} \in \mathcal{M}_{3}(\mathbb{R}) \text{, et } 
b=
\begin{pmatrix}
  2 \\
  2 \\
  8 \\
\end{pmatrix} \in \mathcal{M}_{3, 1}(\mathbb{R})$\vspace{10pt}\\
\underline{Calculons le vecteur $x^{(1)}$ (vecteur x trouvé après 1 itération de l'algorithme) solution du système $Ax=b$, }\vspace{10pt}\\
\underline{en prenant comme point initial $x^{(0)}=(0,0,0)$:} \\
\subsubsection{Résolution par le calcul itératif}\label{calcite}
Dans cette sous-partie, nous résolverons le système de la même manière que le fait l'algorithme sus-cité.\\
Pour obtenir le vecteur $x^{(1)}$ (obtenu à l'itération $k=1$), il nous faut obtenir $x_1^{(1)}$, $x_2^{(1)}$,$x_3^{(1)}$ par la formule suivante :\\
\begin{center}
$x_{i}^{(k+1)}=\frac{1}{a_{i,i}}\Bigg[b_i-\Bigg(\sum \limits_{j=i+1}^{n}a_{i,j}x_j^{(k)} + \sum \limits_{j=1}^{i-1}a_{i,j}x_j^{(k+1)}\Bigg)\Bigg]   \text{  pour  } i=1, ..., 3$
\end{center}
\underline{Pour $i=1$:}\\
\begin{align}
  x_{1}^{(1)}&=\frac{1}{a_{1,1}}\Bigg[b_1-\Bigg(\sum \limits_{j=2}^{3}a_{1,j}x_j^{(0)} + \sum \limits_{j=1}^{0}a_{1,j}x_j^{(1)}\Bigg)\Bigg]\\
  &=\frac{1}{1}\Bigg[2-\Bigg(a_{1,2}x_2^{(0)} + a_{2,2}x_2^{(0)}+0\Bigg)\Bigg]\\
  &=2-2\times 0 -3\times 0=2
\end{align}
\underline{Pour $i=2$:}\\
\begin{align}
  x_{2}^{(1)}&=\frac{1}{a_{2,2}}\Bigg[b_2-\Bigg(\sum \limits_{j=3}^{3}a_{2,j}x_j^{(0)} + \sum \limits_{j=1}^{1}a_{2,j}x_j^{(1)}\Bigg)\Bigg]\\
  &=\frac{1}{3}\Bigg[2-\Bigg(a_{2,3}x_3^{(0)} + a_{2,1}x_1^{(1)}\Bigg)\Bigg]\\
  &=\frac{1}{3}\Bigg(2-3\times0-1\times2\Bigg)\\
  &=\frac{1}{3}\times 0=0
\end{align}
\underline{Pour $i=3$:}\\
\begin{align}
  x_{3}^{(1)}&=\frac{1}{a_{3,3}}\Bigg[b_3-\Bigg(\sum \limits_{j=4}^{3}a_{3,j}x_j^{(0)} + \sum \limits_{j=1}^{2}a_{3,j}x_j^{(1)}\Bigg)\Bigg]\\
  &=\frac{1}{8}\Bigg[8-\Bigg(0+a_{3,1}x_1^{(1)} + a_{3,2}x_2^{(1)}\Bigg)\Bigg]\\
  &=\frac{1}{8}\Bigg(8-3\times 2 - 7\times 0\Bigg)\\
  &=\frac{1}{8}\times 2=\frac{1}{4}
\end{align}
\underline{Conclusion:}\\
Nous avons $x_1^{(1)}=2$, $x_2^{(1)}=0$,  $x_3^{(1)}=\frac{1}{4}$.  Et donc, $x^{(1)}=
\begin{pmatrix}
  x_1^{(1)}\vspace{6pt}\\
  x_2^{(1)}\vspace{6pt}\\
  x_3^{(1)}
\end{pmatrix}=
\begin{pmatrix}
  2\vspace{6pt}\\
  0\vspace{6pt}\\
  \frac{1}{4}
\end{pmatrix}$0
\subsubsection{Résolution par le calcul matriciel}
Dans la section \ref{decompMatrice}, nous avons vu que l'on pouvait décomposer la matrice $A$ par une matrice diagonale $D$, une matrice triangulaire inférieure $E$, et une matrice triangulaire supérieure $F$. Ceci fait, nous pouvons obtenir le vecteur $x$ par la formule suivante: \\
\begin{center}
  $x^{(k+1)}=D^{-1}[b-(E+F)x^{(k)}]$\\
\end{center}
Nous avons alors:\vspace{8pt}\\
$A=
\begin{pmatrix}
  1&2&2\\
  1&3&3\\
  3&7&8\\
\end{pmatrix}=
\underbrace{\begin{pmatrix}
  1&0&0\\
  0&3&0\\
  0&0&8\\
\end{pmatrix}}_{D}-
\underbrace{\begin{pmatrix}
  0&-2&-2\\
  0&0&-3\\
  0&0&0\\
\end{pmatrix}}_{E}-
\underbrace{\begin{pmatrix}
  0&0&0\\
  -1&0&0\\
  -3&-7&0\\
\end{pmatrix}}_{F}
$\vspace{8pt}\\
Nous obtenons alors:\vspace{8pt}\\
\begin{align}
x^{(1)}&=\begin{pmatrix}
  \frac{1}{1}&0&0\\
  0&\frac{1}{3}&0\\
  0&0&\frac{1}{8}\\
\end{pmatrix}\Bigg[\begin{pmatrix}
  2 \\
  2 \\
  8 \\
\end{pmatrix}-\Bigg(\begin{pmatrix}
  0&-2&-2\\
  0&0&-3\\
  0&0&0\\
\end{pmatrix}+\begin{pmatrix}
  0&0&0\\
  -1&0&0\\
  -3&-7&0\\
\end{pmatrix}\Bigg)\begin{pmatrix}
  0 \\
  0 \\
  0 \\
\end{pmatrix}\Bigg]\\
&=\begin{pmatrix}
  \frac{1}{1}&0&0\\
  0&\frac{1}{3}&0\\
  0&0&\frac{1}{8}\\
\end{pmatrix}\begin{pmatrix}
  2 \\
  2 \\
  8 \\
\end{pmatrix}\\
&=\begin{pmatrix}
  2 \vspace{4pt}\\
  \frac{2}{3} \vspace{4pt}\\
  1 \\
\end{pmatrix}
\end{align}
\textit{\underline{Remarque:} Au fur et à mesure des itérations, le vecteur $x$ donné par le calcul itératif effectué dans la partie \ref{calcite} se rapproche de la solution donnée par le précédent calcul. Il est alors normal que le vecteur trouvé au bout de la première itération soit différent du vecteur trouvé ci-dessus.}
\subsection{Implémentation}
\subsubsection{Commentaires fonctionnels}
\textit{\underline{Note:} L'implémentation qui suit utilise exactement les mêmes fonctions usuelles de manipulation de matrice que l'implémentation de l'algorithme de Gauss décrit dans la section \ref{fonctusu}. De plus, dans cette implémentation, nous définirons une variable $k$ qui sera notre compteur d'itération et qui permettra l'arrêt de notre code si la suite ne converge pas. Enfin, notre variable erreur $\varepsilon$ sera mise à jour à chaque itération de la manière suivante:\\}
\begin{center}
  $\varepsilon^{(k)}=p^{(k)}=Max_{i=1,...,n}|\overline{x}_i-\widetilde{x}_i^k|$
\end{center}\vspace{8pt}
Nous détaillerons dans cette section uniquement les fonctions dites "non-usuelles" qui vont nous servir pour l'implémentation de l'algorithme de Gauss-Seidel. Il s'agit ici de la fonction de mise à jour de notre variable erreur et de la fonction implémentant l'algorithme de Gauss-Seidel.\vspace{8pt}\\
\textbf{\underline{Fonction \textit{majEpsilon:}}}\vspace{6pt}\\
La fonction \textbf{majEpsilon} permet de mettre à jour la variable d'erreur $\varepsilon$ lors de l'exécution de notre algorithme. Grâce au vecteur $x^{(k)}$ qui représente la solution actuelle de notre système d'équations, la fonction calcule la différence absolue entre chaque élément $x_i^{(k)}$ et $1$. Cela permet de mesurer à quel point les valeurs actuelles se rapprochent de 1, qui est notre valeur cible pour les solutions convergentes. La fonction conserve le maximum de ces différences absolues en tant que mesure d'erreur afin de mettre à jour notre variable $\varepsilon$. Cela permet de contrôler la précision de l'algorithme et de décider quand il a convergé de manière satisfaisante vers la solution recherchée. La fonction \textbf{majEpsilon} joue donc un rôle dans la détermination du critère d'arrêt de l'algorithme. Voici son implémentation en C:
\begin{lstlisting}[language=C,inputencoding=utf8, basicstyle=\fontsize{8}{10}\selectfont]
float majEpsilon(float** matXk, int row){
  float maxforEps=0;
  for(int i=0; i<row; i++){
      float soustr=fabs(1-matXk[i][0]);
      printf("%f\n", matXk[i][0]);
      if (soustr>maxforEps){
          maxforEps=soustr;
      }
  }
  return maxforEps;
}
\end{lstlisting}
\textbf{\underline{Fonction \textit{gaussSeidel:}}}\vspace{6pt}\\
La fonction \textbf{gaussSeidel} implémente l'algorithme de Gauss-Seidel tel que décrit précédemment dans la section \ref{algogs}. Par la programmation itérative, notre algorithme mettra à jour les solutions actuelles jusqu'à ce que l'erreur minimale définie soit atteinte ou que le nombre maximal d'itérations soit atteint. Voici son implémentation en C:
\begin{lstlisting}[mathescape=true, language=C,inputencoding=utf8, basicstyle=\fontsize{8}{10}\selectfont]
  float** gaussSeidel(float **matA, float **matB, int row, int column, int nbIterMax){
    //CREATING OUR INITIAL VECTOR
    float **matXk=createMatrix(row, 1);
    for(int rowX=0; rowX<row; rowX++){
			matXk[rowX][0]=0;
    }

    //CREATING OUR SOLUTION VECTOR AT ITERATION k
    float **matXk1=createMatrix(row, 1);

    //INITIALIZING OUR ERROR VARIABLE AND ITERATION COUNTER
    float epsilon=majEpsilon(matXk, row);;
    int iter=0;
    
    while ((epsilon>=pow(10,-6)) && (iter<nbIterMax)){
        for(int i=0; i<row ; i++){
            float sumF=0;
            float sumE=0;

            //CALCULATION OF F
            for(int j=i+1; j<row ; j++){
                sumF+=matA[i][j]*matXk[j][0];
            }

            //CALCULATION OF E
            for(int j=0; j<i ; j++){
                sumE+=matA[i][j]*matXk1[j][0];
            }

            //CALCULATION OF ELEMENT $X_i^{(k)}$"
            matXk1[i][0]=(matB[i][0]-sumF-sumE)/matA[i][i];

        }

        //UPDATING OUR SOLUTION VECTOR
        matXk[0][0]=matXk1[0][0];
        matXk[1][0]=matXk1[1][0];
        matXk[2][0]=matXk1[2][0];

        //UPDATING OUR ERROR VARIABLE AND ITERATION COUNTER
        epsilon=majEpsilon(matXk, row);
        iter+=1;
    }

    //RETURN THE SOLUTION VECTOR
    return matXk1;
}
\end{lstlisting}
\subsection{Exemples d'exécution}
\end{document}


\documentclass{report}
\setlength{\headheight}{24.1638pt}
%packages
\usepackage[french]{babel}
\usepackage[T1]{fontenc}
\usepackage[utf8]{inputenc}
\usepackage{mathtools}
\usepackage{amssymb}
\usepackage{hyperref}
\usepackage{float}
\usepackage{amsthm}
\usepackage{listings}
\usepackage{geometry}
\usepackage{setspace}
\usepackage{graphicx}
\usepackage{fancyhdr}
\usepackage{subcaption}
\usepackage{cleveref}

%commands
\newtheorem{defi}{Définition}
\renewcommand{\thedefi}{\empty{}}

\renewcommand\headrulewidth{1pt}
\newcommand{\crule}[3][c]{%
    \par\noindent
    \makebox[\linewidth][#1]{\rule{#2\linewidth}{#3}}}
\renewcommand{\thechapter}{\Roman{chapter}}

%Style de page
\pagestyle{fancy}
\fancyhead[L]{}
\fancyhead[C]{}
\fancyhead[R]{\leftmark}
\allowdisplaybreaks
\geometry{hmargin=3cm,vmargin=2.5cm}


%préambule
\begin{document}
\section{Méthode de Jacobi}
Rappelons que la méthode de \textbf{Jacobi} est itérative et ne garantit pas toujours un résultat. La méthode est définie si A est définie positive.\\
L'algorithme permet de trouver un résultat si la matrice est dites à diagonale strictement dominante. \\
Autrement dit, Soit $[a_{ij}]_{0 \leq i,j \leq n}$ les coefficients réels peuplant $A \in \mathcal{M}_{n,n} (\mathbb{R})$, alors si:\\
$\forall i, \vert a_{i,i} \vert <  \sum \limits_{i \neq j} \vert a_{ij} \vert$, on a que Jacobi converge vers l'unique solution du système $Ax=b$. 
\subsection{Principe de la méthode}
On veut résoudre $Ax=b$ avec $A \in \mathcal{M}_{n,n} (\mathbb{R})$, $n \in \mathbb{N}$, $x$ la vecteur colonne contenant les inconnus et $b$ le vecteur colonne des solution. \\
On pose $D \in \mathcal{M}_{n,n}(\mathbb{R})$ la matrice contenant les coefficients $[a_{i,j}]_{0 \leq i=j \leq n}$ de $A$. \\
On pose aussi $E$ et $F$ avec $E$ la matrice triangulaire opposée inférieure de $A$ et $F$ la matrice supérieure opposée de $A$.  \\
On obtient alors:
\begin{align}
Ax &=b \\
(D-E-F)x &= b \\
Dx -(E+F)x &= b \\
x &=D^{-1}(E+F)x+D^{-1}b \\
x^{k+1} &=D^{-1}(E+F)x^{k}+D^{-1}b 
\end{align}
Ce qui donne l'algorithme suivant: \\
Soit $\epsilon$ L'erreur maximale, un point initial $x^0$ et $k=0$ \\
avec $\epsilon^{0} = \vert \vert A x^0 -b \vert \vert$ \\
On obtient: 
\begin{lstlisting}[mathescape=true, frame=single]
Tant que ($\epsilon^{(k)} \leq \epsilon$)
$x^{k+1}_i = \frac{1}{a_{ii}} [ b_i - \sum \limits_{j\neq i} a_{ij}x_j^{(k)}], i=1,\ldots,n$
$\epsilon^{k+1} = \vert \vert Ax^{k+1} -b \vert \vert$
$k=k+1$
FIN JACOBI
\end{lstlisting}
\textbf{Remarque, on ajoutera aussi un nombre d'iterations maximum afin de ne pas être dans le cas d'une boucle infinies (si jacobi diverge alors l'erreur augmente)}.
\subsection{Résolution manuelle}
\textit{Nous en détaillerons seulement une itération} \\
Soit
$A= \begin{pmatrix}
4 & 1 & 0 \\
-1 & 3 & 6 \\
-2 & -5 & -3
\end{pmatrix}$,
$b = \begin{pmatrix}
8 \\
3 \\
8 \\
\end{pmatrix}$
,$x= \begin{pmatrix}
x_1 \\
x_2 \\
x_3
\end{pmatrix}$ 
et $x^{(0)} = \begin{pmatrix}
0 \\
0\\
0\\
\end{pmatrix}$ \\
On a $A = \underbrace{\begin{pmatrix}
4 & 0 & 0 \\
0 & 3 & 0 \\
0 & 0 & -2
\end{pmatrix}}_{D}
- 
\underbrace{\begin{pmatrix}
0 & 0& 0 \\
1 & 0 & 0 \\
2 & 5 &0 
\end{pmatrix}}_E
- \underbrace{\begin{pmatrix}
0 & -1 & 0 \\
0 & 0 & -6\\
0 & 0 & 0
\end{pmatrix}}_F$ \\
On a donc $x^{k+1}= D^{-1} [(E+F)x^{k} + b]$ \\
Dans le cas présent on obtient alors: \\
$x^1 = \begin{pmatrix}
\frac{1}{4} & 0 & 0 \\
0 & \frac{1}{3} & 0 \\
0 & 0 & -\frac{1}{2}
\end{pmatrix}
\left[ \left[ \begin{pmatrix}
0 & 0 & 0 \\
1 & 0  & 0 \\
2 & 5 & 0 \\
\end{pmatrix} + 
\begin{pmatrix}
0 & -1 & 0 \\
0 & 0 & -6 \\
0 & 0 & 0 
\end{pmatrix}
\right]
\begin{pmatrix}
0\\
0\\
0
\end{pmatrix}
+
\begin{pmatrix}
8 \\
3\\
8
\end{pmatrix}
  \right]$ \\
$x^1 = \begin{pmatrix}
2 \\
1 \\
-4
\end{pmatrix}$ \\
et
\begin{align*}
\epsilon^{(1)} & = \vert \vert Ax^{(1)} -b \vert \vert \\
\epsilon^{(1)} & = \vert \vert \begin{pmatrix}
4 & 1 & 0 \\
-1 & 3 & 6 \\
-2 & -5 & -3
\end{pmatrix}
\begin{pmatrix}
2\\
1\\
-4
\end{pmatrix} - 
\begin{pmatrix}
8 \\
3 \\
8
\end{pmatrix} \vert \vert \\
\epsilon^{(1)} & = \vert \vert \begin{pmatrix}
1 \\
-26 \\
-5
\end{pmatrix} \vert \vert \\
\epsilon^{(1)} & =  \sqrt{1^2+(-26)^2+(-5)^2} \\
\epsilon^{(1)} & =  \sqrt{(702)} \\
\end{align*}

\subsection{Implémentation}
Pour l'implémentation de cette méthode, nous utiliserons $\epsilon$ comme suit: \\
$\epsilon^{(k)} = p^k = \text{Max}_{i=1,\ldots, n} \vert \bar{x_i} -\tilde{x_i}^k \vert $ \\
Où $\bar{x_i}$ est les résultat attendu et $\tilde{x_i}^k$ est l'approximation trouvée à l'étape $k$. \\
De plus on utilisera aussi une limite d'occurrence, pour pouvoir gérer les matrices où \textbf{Jacobi} diverge.\\
En l'occurrence on se fixera un $\epsilon = 10^{-6}$ et un nombre d'itération maximum fixé à $1000$
\subsection{Code}
\textit{On ne détaillera ici seulement les fonctions dites "non triviales" c'est-à-dire les fonctions étant en lien directe avec l'algorithme décrit.\\De plus chaque fonction présentée sera dépouillée de toute fonction d'affichage permettant de produire des chiffres liés à l'utilisation du programme (pour une question de lisibilité}
\begin{lstlisting}[language=C,inputencoding=utf8, basicstyle=\fontsize{8}{10}\selectfont,caption=jacobi.c]
int jacobi(float **A, float *vector, float *b, float *S, int n, float minErr,int bound) {
  float epsilon = epsi(S, vector, n);
  int k = 0;
  while (k < bound && epsilon >= minErr) {
    for (int i = 0; i < n; i++) {
      vector[i] = (1 / A[i][i]) * (b[i] - jacobiSum(A, vector, n, i));
    }
    epsilon = epsi(vector, S, n);
    k++;
  }
  return k;
}
\end{lstlisting}
\textbf{Commentaires:} Nous remarquerons que j'implémentation de l'algorithme de Jacobi est similaire à ce que a été présenté ultérieurement. Pour des questions de lisibilité, $\epsilon^{(k)}$ est calculée par la fonction \textit{epsi()}, de même pour \textbf{jacobiSum()}, le calcul a été séparé afin de faciliter la comprehension du programme.
\begin{lstlisting}[language=C,inputencoding=utf8, basicstyle=\fontsize{8}{10}\selectfont,caption=jacobiSum() function in "source.h"]
float jacobiSum(float **A, float *V, int m, int i) {
  float s = 0;
  for (int j = 0; j < m; j++) {
    if (j != i)
      s += A[i][j] * V[j];
  }
  return s;
}
\end{lstlisting}
\textbf{Commentaire:} Permet de calculer de façon lisible est clair ceci: \\
\begin{center}
$\sum \limits_{j \neq i} a_{ij}x_j^{(k)}, i=1,\ldots k$
\end{center}
\begin{lstlisting}[language=C,inputencoding=utf8, basicstyle=\fontsize{8}{10}\selectfont,caption=epsi() function in "source.h"]
float epsi(float *V, float *S, int n) {
  float max = Fabs(V[0] - S[0]);
  for (int i = 1; i < n; i++) {
    if (Fabs(S[i] - V[i]) > max)
      max = Fabs(S[i] - V[i]);
  }
  return max;
}
\end{lstlisting}
\textbf{Commentaires:} Utilise la fonction \textit{Fabs()} que nous avons implémenter, elle renvoie la valeur absolue d'un nombre flottant.\\
Cette fonction \textit{epsi()} permet donc de calculer l'erreur entre deux itération de la façon suivante: \\
\begin{center}
$\epsilon^{(k)} = \text{Max}_{i=1,\ldots, n} \vert \bar{x_i} -\tilde{x_i}^k \vert $ \\
\end{center}
\begin{lstlisting}[language=C,inputencoding=utf8, basicstyle=\fontsize{8}{10}\selectfont,caption=conv() function in "source.h"]
int conv(float **A, int m) {
  float sl = 0;
  for (int i = 0; i < m; i++) {
    for (int j = 0; j < m; j++) {
      if (i != j)
        sl += fabs(A[i][j]);
    }
    if (A[i][i] - sl <= 0)
      return 0;
  }
  return 1;
}
\end{lstlisting}
\textbf{Commentaire} \textit{conv()} est une fonction utilitaire permettant d'effectuer une prediction sur la convergeance potentielle d'une matrice par jacobi.\\
Pour cela on vérifie si la matrice sur laquelle on effectue jacobi est à diagonale strictemennt dominante. \\
Ce qui se vérifie par cette formule: \\
\begin{center}
$\forall i, \vert a_{ii} \vert > \sum \limits_{j\neq i} \vert a_{ij} \vert$
\end{center}
\subsection{Entrées / Sorties}
\subsubsection{Entrées}
Le programme prend en entrée 2 paramètres soient 
\begin{enumerate}
\item minErr, ou l'erreur minimale tolérée
\item bounds, qui désigne la limite d'occurrence du programme (en cas de divergeance)
\end{enumerate}
L'utilisateur peut ensuite décider de remplir manuellement sa matrice où de rediriger le flux d'un fichier comme suit:
\begin{lstlisting}[language=C,inputencoding=utf8, basicstyle=\fontsize{8}{10}\selectfont,caption=A4.txt]
3 3
7 6 9
4 5 -4
-7 -3 8 
\end{lstlisting}
Avec sur la première ligne: les dimensions de la matrice suivit, sur les ligne suivantes, des coefficients de la matrice
\subsection{Sorties}
Le flux d'erreur sera réservé afin de produire des données engendrant des graphiques (des données formatées). Il s'agit de l'énumération de tout les $\epsilon$ calculés suivit du nombre d'itération. \\
Sur les autre lignes seront inscrit des messages facilitant la prise en main du programme pour l'utilisateur. \\
Il figuera ensuite la prédiction quant à la convergence de la méthode.\\
Enfin la dernière lignes représentera le nombre de $\epsilon$ calculés. \\
\subsection{Exécution}
Voici l'exécution sur les 12 matrices de la méthode de jacobi.\\
\textit{La redirection de flux de sortie du programme est la suivante: >}
\begin{lstlisting}[language=C,inputencoding=utf8, basicstyle=\fontsize{8}{10}\selectfont,caption=Execution with A1 matrix]
A matrix dimensions: 
Enter coefficient for 0x55bcc971d300[0][0]Enter coefficient for 0x55bcc971d300[0][1]Enter coefficient for 0x55bcc971d300[0][2]Enter coefficient for 0x55bcc971d300[1][0]Enter coefficient for 0x55bcc971d300[1][1]Enter coefficient for 0x55bcc971d300[1][2]Enter coefficient for 0x55bcc971d300[2][0]Enter coefficient for 0x55bcc971d300[2][1]Enter coefficient for 0x55bcc971d300[2][2]
 ===== CONVERGENCE PREDICTION: may not conv =====
PRINTING VECTOR FROM: 0x55bcc971d2e0 LOCATION :
-nan -nan -nan
EPSILON CALCULATED 191
\end{lstlisting}
\begin{lstlisting}[language=C,inputencoding=utf8, basicstyle=\fontsize{8}{10}\selectfont,caption=Execution with A2 matrix]
A matrix dimensions: 
Enter coefficient for 0x5618a8a9b300[0][0]Enter coefficient for 0x5618a8a9b300[0][1]Enter coefficient for 0x5618a8a9b300[0][2]Enter coefficient for 0x5618a8a9b300[1][0]Enter coefficient for 0x5618a8a9b300[1][1]Enter coefficient for 0x5618a8a9b300[1][2]Enter coefficient for 0x5618a8a9b300[2][0]Enter coefficient for 0x5618a8a9b300[2][1]Enter coefficient for 0x5618a8a9b300[2][2]
 ===== CONVERGENCE PREDICTION: may not conv =====
PRINTING VECTOR FROM: 0x5618a8a9b2e0 LOCATION :
-nan -nan -nan
EPSILON CALCULATED 127
\end{lstlisting}
\begin{lstlisting}[language=C,inputencoding=utf8, basicstyle=\fontsize{8}{10}\selectfont,caption=Execution with A3 matrix]
A matrix dimensions: 
Enter coefficient for 0x558cd60e7300[0][0]Enter coefficient for 0x558cd60e7300[0][1]Enter coefficient for 0x558cd60e7300[0][2]Enter coefficient for 0x558cd60e7300[1][0]Enter coefficient for 0x558cd60e7300[1][1]Enter coefficient for 0x558cd60e7300[1][2]Enter coefficient for 0x558cd60e7300[2][0]Enter coefficient for 0x558cd60e7300[2][1]Enter coefficient for 0x558cd60e7300[2][2]
 ===== CONVERGENCE PREDICTION: may not conv =====
PRINTING VECTOR FROM: 0x558cd60e72e0 LOCATION :
1.000000 1.000000 1.000000
EPSILON CALCULATED 8
\end{lstlisting}
\begin{lstlisting}[language=C,inputencoding=utf8, basicstyle=\fontsize{8}{10}\selectfont,caption=Execution with A4 matrix]
A matrix dimensions: 
Enter coefficient for 0x5634c961e300[0][0]Enter coefficient for 0x5634c961e300[0][1]Enter coefficient for 0x5634c961e300[0][2]Enter coefficient for 0x5634c961e300[1][0]Enter coefficient for 0x5634c961e300[1][1]Enter coefficient for 0x5634c961e300[1][2]Enter coefficient for 0x5634c961e300[2][0]Enter coefficient for 0x5634c961e300[2][1]Enter coefficient for 0x5634c961e300[2][2]
 ===== CONVERGENCE PREDICTION: may not conv =====
PRINTING VECTOR FROM: 0x5634c961e2e0 LOCATION :
1.000000 1.000000 1.000000
EPSILON CALCULATED 58
\end{lstlisting}
\begin{lstlisting}[language=C,inputencoding=utf8, basicstyle=\fontsize{8}{10}\selectfont,caption=Execution with A5 dimensions: 3x3]
A matrix dimensions: 
Enter coefficient for 0x5617a96a0300[0][0]Enter coefficient for 0x5617a96a0300[0][1]Enter coefficient for 0x5617a96a0300[0][2]Enter coefficient for 0x5617a96a0300[1][0]Enter coefficient for 0x5617a96a0300[1][1]Enter coefficient for 0x5617a96a0300[1][2]Enter coefficient for 0x5617a96a0300[2][0]Enter coefficient for 0x5617a96a0300[2][1]Enter coefficient for 0x5617a96a0300[2][2]
 ===== CONVERGENCE PREDICTION: may not conv =====
PRINTING VECTOR FROM: 0x5617a96a02e0 LOCATION :
1.000001 1.000000 1.000000
EPSILON CALCULATED 13 
\end{lstlisting}
\begin{lstlisting}[language=C,inputencoding=utf8, basicstyle=\fontsize{8}{10}\selectfont,caption=Execution with A5 dimensions: 6x6]
A matrix dimensions: 
Enter coefficient for 0x560e22b20300[0][0]Enter coefficient for 0x560e22b20300[0][1]Enter coefficient for 0x560e22b20300[0][2]Enter coefficient for 0x560e22b20300[0][3]Enter coefficient for 0x560e22b20300[0][4]Enter coefficient for 0x560e22b20300[0][5]Enter coefficient for 0x560e22b20300[1][0]Enter coefficient for 0x560e22b20300[1][1]Enter coefficient for 0x560e22b20300[1][2]Enter coefficient for 0x560e22b20300[1][3]Enter coefficient for 0x560e22b20300[1][4]Enter coefficient for 0x560e22b20300[1][5]Enter coefficient for 0x560e22b20300[2][0]Enter coefficient for 0x560e22b20300[2][1]Enter coefficient for 0x560e22b20300[2][2]Enter coefficient for 0x560e22b20300[2][3]Enter coefficient for 0x560e22b20300[2][4]Enter coefficient for 0x560e22b20300[2][5]Enter coefficient for 0x560e22b20300[3][0]Enter coefficient for 0x560e22b20300[3][1]Enter coefficient for 0x560e22b20300[3][2]Enter coefficient for 0x560e22b20300[3][3]Enter coefficient for 0x560e22b20300[3][4]Enter coefficient for 0x560e22b20300[3][5]Enter coefficient for 0x560e22b20300[4][0]Enter coefficient for 0x560e22b20300[4][1]Enter coefficient for 0x560e22b20300[4][2]Enter coefficient for 0x560e22b20300[4][3]Enter coefficient for 0x560e22b20300[4][4]Enter coefficient for 0x560e22b20300[4][5]Enter coefficient for 0x560e22b20300[5][0]Enter coefficient for 0x560e22b20300[5][1]Enter coefficient for 0x560e22b20300[5][2]Enter coefficient for 0x560e22b20300[5][3]Enter coefficient for 0x560e22b20300[5][4]Enter coefficient for 0x560e22b20300[5][5]
 ===== CONVERGENCE PREDICTION: may not conv =====
PRINTING VECTOR FROM: 0x560e22b202e0 LOCATION :
1.000001 1.000000 1.000000 1.000000 1.000000 1.000000
EPSILON CALCULATED 14  
\end{lstlisting}
\begin{lstlisting}[language=C,inputencoding=utf8, basicstyle=\fontsize{8}{10}\selectfont,caption=Execution with A5 dimensions: 8x8]
A matrix dimensions: 
Enter coefficient for 0x558ffc992320[0][0]Enter coefficient for 0x558ffc992320[0][1]Enter coefficient for 0x558ffc992320[0][2]Enter coefficient for 0x558ffc992320[0][3]Enter coefficient for 0x558ffc992320[0][4]Enter coefficient for 0x558ffc992320[0][5]Enter coefficient for 0x558ffc992320[0][6]Enter coefficient for 0x558ffc992320[0][7]Enter coefficient for 0x558ffc992320[1][0]Enter coefficient for 0x558ffc992320[1][1]Enter coefficient for 0x558ffc992320[1][2]Enter coefficient for 0x558ffc992320[1][3]Enter coefficient for 0x558ffc992320[1][4]Enter coefficient for 0x558ffc992320[1][5]Enter coefficient for 0x558ffc992320[1][6]Enter coefficient for 0x558ffc992320[1][7]Enter coefficient for 0x558ffc992320[2][0]Enter coefficient for 0x558ffc992320[2][1]Enter coefficient for 0x558ffc992320[2][2]Enter coefficient for 0x558ffc992320[2][3]Enter coefficient for 0x558ffc992320[2][4]Enter coefficient for 0x558ffc992320[2][5]Enter coefficient for 0x558ffc992320[2][6]Enter coefficient for 0x558ffc992320[2][7]Enter coefficient for 0x558ffc992320[3][0]Enter coefficient for 0x558ffc992320[3][1]Enter coefficient for 0x558ffc992320[3][2]Enter coefficient for 0x558ffc992320[3][3]Enter coefficient for 0x558ffc992320[3][4]Enter coefficient for 0x558ffc992320[3][5]Enter coefficient for 0x558ffc992320[3][6]Enter coefficient for 0x558ffc992320[3][7]Enter coefficient for 0x558ffc992320[4][0]Enter coefficient for 0x558ffc992320[4][1]Enter coefficient for 0x558ffc992320[4][2]Enter coefficient for 0x558ffc992320[4][3]Enter coefficient for 0x558ffc992320[4][4]Enter coefficient for 0x558ffc992320[4][5]Enter coefficient for 0x558ffc992320[4][6]Enter coefficient for 0x558ffc992320[4][7]Enter coefficient for 0x558ffc992320[5][0]Enter coefficient for 0x558ffc992320[5][1]Enter coefficient for 0x558ffc992320[5][2]Enter coefficient for 0x558ffc992320[5][3]Enter coefficient for 0x558ffc992320[5][4]Enter coefficient for 0x558ffc992320[5][5]Enter coefficient for 0x558ffc992320[5][6]Enter coefficient for 0x558ffc992320[5][7]Enter coefficient for 0x558ffc992320[6][0]Enter coefficient for 0x558ffc992320[6][1]Enter coefficient for 0x558ffc992320[6][2]Enter coefficient for 0x558ffc992320[6][3]Enter coefficient for 0x558ffc992320[6][4]Enter coefficient for 0x558ffc992320[6][5]Enter coefficient for 0x558ffc992320[6][6]Enter coefficient for 0x558ffc992320[6][7]Enter coefficient for 0x558ffc992320[7][0]Enter coefficient for 0x558ffc992320[7][1]Enter coefficient for 0x558ffc992320[7][2]Enter coefficient for 0x558ffc992320[7][3]Enter coefficient for 0x558ffc992320[7][4]Enter coefficient for 0x558ffc992320[7][5]Enter coefficient for 0x558ffc992320[7][6]Enter coefficient for 0x558ffc992320[7][7]
 ===== CONVERGENCE PREDICTION: may not conv =====
PRINTING VECTOR FROM: 0x558ffc9922f0 LOCATION :
1.000001 1.000000 1.000000 1.000000 1.000000 1.000000 1.000000 1.000000
EPSILON CALCULATED 14:       
\end{lstlisting}
\begin{lstlisting}[language=C,inputencoding=utf8, basicstyle=\fontsize{8}{10}\selectfont,caption=Execution with A5 dimensions: 10x10]
A matrix dimensions: 
Enter coefficient for 0x5625d23be320[0][0]Enter coefficient for 0x5625d23be320[0][1]Enter coefficient for 0x5625d23be320[0][2]Enter coefficient for 0x5625d23be320[0][3]Enter coefficient for 0x5625d23be320[0][4]Enter coefficient for 0x5625d23be320[0][5]Enter coefficient for 0x5625d23be320[0][6]Enter coefficient for 0x5625d23be320[0][7]Enter coefficient for 0x5625d23be320[0][8]Enter coefficient for 0x5625d23be320[0][9]Enter coefficient for 0x5625d23be320[1][0]Enter coefficient for 0x5625d23be320[1][1]Enter coefficient for 0x5625d23be320[1][2]Enter coefficient for 0x5625d23be320[1][3]Enter coefficient for 0x5625d23be320[1][4]Enter coefficient for 0x5625d23be320[1][5]Enter coefficient for 0x5625d23be320[1][6]Enter coefficient for 0x5625d23be320[1][7]Enter coefficient for 0x5625d23be320[1][8]Enter coefficient for 0x5625d23be320[1][9]Enter coefficient for 0x5625d23be320[2][0]Enter coefficient for 0x5625d23be320[2][1]Enter coefficient for 0x5625d23be320[2][2]Enter coefficient for 0x5625d23be320[2][3]Enter coefficient for 0x5625d23be320[2][4]Enter coefficient for 0x5625d23be320[2][5]Enter coefficient for 0x5625d23be320[2][6]Enter coefficient for 0x5625d23be320[2][7]Enter coefficient for 0x5625d23be320[2][8]Enter coefficient for 0x5625d23be320[2][9]Enter coefficient for 0x5625d23be320[3][0]Enter coefficient for 0x5625d23be320[3][1]Enter coefficient for 0x5625d23be320[3][2]Enter coefficient for 0x5625d23be320[3][3]Enter coefficient for 0x5625d23be320[3][4]Enter coefficient for 0x5625d23be320[3][5]Enter coefficient for 0x5625d23be320[3][6]Enter coefficient for 0x5625d23be320[3][7]Enter coefficient for 0x5625d23be320[3][8]Enter coefficient for 0x5625d23be320[3][9]Enter coefficient for 0x5625d23be320[4][0]Enter coefficient for 0x5625d23be320[4][1]Enter coefficient for 0x5625d23be320[4][2]Enter coefficient for 0x5625d23be320[4][3]Enter coefficient for 0x5625d23be320[4][4]Enter coefficient for 0x5625d23be320[4][5]Enter coefficient for 0x5625d23be320[4][6]Enter coefficient for 0x5625d23be320[4][7]Enter coefficient for 0x5625d23be320[4][8]Enter coefficient for 0x5625d23be320[4][9]Enter coefficient for 0x5625d23be320[5][0]Enter coefficient for 0x5625d23be320[5][1]Enter coefficient for 0x5625d23be320[5][2]Enter coefficient for 0x5625d23be320[5][3]Enter coefficient for 0x5625d23be320[5][4]Enter coefficient for 0x5625d23be320[5][5]Enter coefficient for 0x5625d23be320[5][6]Enter coefficient for 0x5625d23be320[5][7]Enter coefficient for 0x5625d23be320[5][8]Enter coefficient for 0x5625d23be320[5][9]Enter coefficient for 0x5625d23be320[6][0]Enter coefficient for 0x5625d23be320[6][1]Enter coefficient for 0x5625d23be320[6][2]Enter coefficient for 0x5625d23be320[6][3]Enter coefficient for 0x5625d23be320[6][4]Enter coefficient for 0x5625d23be320[6][5]Enter coefficient for 0x5625d23be320[6][6]Enter coefficient for 0x5625d23be320[6][7]Enter coefficient for 0x5625d23be320[6][8]Enter coefficient for 0x5625d23be320[6][9]Enter coefficient for 0x5625d23be320[7][0]Enter coefficient for 0x5625d23be320[7][1]Enter coefficient for 0x5625d23be320[7][2]Enter coefficient for 0x5625d23be320[7][3]Enter coefficient for 0x5625d23be320[7][4]Enter coefficient for 0x5625d23be320[7][5]Enter coefficient for 0x5625d23be320[7][6]Enter coefficient for 0x5625d23be320[7][7]Enter coefficient for 0x5625d23be320[7][8]Enter coefficient for 0x5625d23be320[7][9]Enter coefficient for 0x5625d23be320[8][0]Enter coefficient for 0x5625d23be320[8][1]Enter coefficient for 0x5625d23be320[8][2]Enter coefficient for 0x5625d23be320[8][3]Enter coefficient for 0x5625d23be320[8][4]Enter coefficient for 0x5625d23be320[8][5]Enter coefficient for 0x5625d23be320[8][6]Enter coefficient for 0x5625d23be320[8][7]Enter coefficient for 0x5625d23be320[8][8]Enter coefficient for 0x5625d23be320[8][9]Enter coefficient for 0x5625d23be320[9][0]Enter coefficient for 0x5625d23be320[9][1]Enter coefficient for 0x5625d23be320[9][2]Enter coefficient for 0x5625d23be320[9][3]Enter coefficient for 0x5625d23be320[9][4]Enter coefficient for 0x5625d23be320[9][5]Enter coefficient for 0x5625d23be320[9][6]Enter coefficient for 0x5625d23be320[9][7]Enter coefficient for 0x5625d23be320[9][8]Enter coefficient for 0x5625d23be320[9][9]
 ===== CONVERGENCE PREDICTION: may not conv =====
PRINTING VECTOR FROM: 0x5625d23be2f0 LOCATION :
1.000001 1.000000 1.000000 1.000000 1.000000 1.000000 1.000000 1.000000 1.000000 1.000000
EPSILON CALCULATED 14
\end{lstlisting}
\begin{lstlisting}[language=C,inputencoding=utf8, basicstyle=\fontsize{8}{10}\selectfont,caption=Execution with A6 dimensions: 3x3]
A matrix dimensions: 
Enter coefficient for 0x55b6d687c300[0][0]Enter coefficient for 0x55b6d687c300[0][1]Enter coefficient for 0x55b6d687c300[0][2]Enter coefficient for 0x55b6d687c300[1][0]Enter coefficient for 0x55b6d687c300[1][1]Enter coefficient for 0x55b6d687c300[1][2]Enter coefficient for 0x55b6d687c300[2][0]Enter coefficient for 0x55b6d687c300[2][1]Enter coefficient for 0x55b6d687c300[2][2]
 ===== CONVERGENCE PREDICTION: may not conv =====
PRINTING VECTOR FROM: 0x55b6d687c2e0 LOCATION :
1.000000 1.000000 1.000000
EPSILON CALCULATED 18
\end{lstlisting}
\begin{lstlisting}[language=C,inputencoding=utf8, basicstyle=\fontsize{8}{10}\selectfont,caption=Execution with A6 dimensions: 6x6]
A matrix dimensions: 
Enter coefficient for 0x55b7a7e47300[0][0]Enter coefficient for 0x55b7a7e47300[0][1]Enter coefficient for 0x55b7a7e47300[0][2]Enter coefficient for 0x55b7a7e47300[0][3]Enter coefficient for 0x55b7a7e47300[0][4]Enter coefficient for 0x55b7a7e47300[0][5]Enter coefficient for 0x55b7a7e47300[1][0]Enter coefficient for 0x55b7a7e47300[1][1]Enter coefficient for 0x55b7a7e47300[1][2]Enter coefficient for 0x55b7a7e47300[1][3]Enter coefficient for 0x55b7a7e47300[1][4]Enter coefficient for 0x55b7a7e47300[1][5]Enter coefficient for 0x55b7a7e47300[2][0]Enter coefficient for 0x55b7a7e47300[2][1]Enter coefficient for 0x55b7a7e47300[2][2]Enter coefficient for 0x55b7a7e47300[2][3]Enter coefficient for 0x55b7a7e47300[2][4]Enter coefficient for 0x55b7a7e47300[2][5]Enter coefficient for 0x55b7a7e47300[3][0]Enter coefficient for 0x55b7a7e47300[3][1]Enter coefficient for 0x55b7a7e47300[3][2]Enter coefficient for 0x55b7a7e47300[3][3]Enter coefficient for 0x55b7a7e47300[3][4]Enter coefficient for 0x55b7a7e47300[3][5]Enter coefficient for 0x55b7a7e47300[4][0]Enter coefficient for 0x55b7a7e47300[4][1]Enter coefficient for 0x55b7a7e47300[4][2]Enter coefficient for 0x55b7a7e47300[4][3]Enter coefficient for 0x55b7a7e47300[4][4]Enter coefficient for 0x55b7a7e47300[4][5]Enter coefficient for 0x55b7a7e47300[5][0]Enter coefficient for 0x55b7a7e47300[5][1]Enter coefficient for 0x55b7a7e47300[5][2]Enter coefficient for 0x55b7a7e47300[5][3]Enter coefficient for 0x55b7a7e47300[5][4]Enter coefficient for 0x55b7a7e47300[5][5]
 ===== CONVERGENCE PREDICTION: may not conv =====
PRINTING VECTOR FROM: 0x55b7a7e472e0 LOCATION :
1.000000 1.000000 0.999999 0.999999 0.999999 1.000000
EPSILON CALCULATED 43
\end{lstlisting}
\begin{lstlisting}[language=C,inputencoding=utf8, basicstyle=\fontsize{8}{10}\selectfont,caption=Execution with A6 dimensions: 8x8]
A matrix dimensions: 
Enter coefficient for 0x5613b3702320[0][0]Enter coefficient for 0x5613b3702320[0][1]Enter coefficient for 0x5613b3702320[0][2]Enter coefficient for 0x5613b3702320[0][3]Enter coefficient for 0x5613b3702320[0][4]Enter coefficient for 0x5613b3702320[0][5]Enter coefficient for 0x5613b3702320[0][6]Enter coefficient for 0x5613b3702320[0][7]Enter coefficient for 0x5613b3702320[1][0]Enter coefficient for 0x5613b3702320[1][1]Enter coefficient for 0x5613b3702320[1][2]Enter coefficient for 0x5613b3702320[1][3]Enter coefficient for 0x5613b3702320[1][4]Enter coefficient for 0x5613b3702320[1][5]Enter coefficient for 0x5613b3702320[1][6]Enter coefficient for 0x5613b3702320[1][7]Enter coefficient for 0x5613b3702320[2][0]Enter coefficient for 0x5613b3702320[2][1]Enter coefficient for 0x5613b3702320[2][2]Enter coefficient for 0x5613b3702320[2][3]Enter coefficient for 0x5613b3702320[2][4]Enter coefficient for 0x5613b3702320[2][5]Enter coefficient for 0x5613b3702320[2][6]Enter coefficient for 0x5613b3702320[2][7]Enter coefficient for 0x5613b3702320[3][0]Enter coefficient for 0x5613b3702320[3][1]Enter coefficient for 0x5613b3702320[3][2]Enter coefficient for 0x5613b3702320[3][3]Enter coefficient for 0x5613b3702320[3][4]Enter coefficient for 0x5613b3702320[3][5]Enter coefficient for 0x5613b3702320[3][6]Enter coefficient for 0x5613b3702320[3][7]Enter coefficient for 0x5613b3702320[4][0]Enter coefficient for 0x5613b3702320[4][1]Enter coefficient for 0x5613b3702320[4][2]Enter coefficient for 0x5613b3702320[4][3]Enter coefficient for 0x5613b3702320[4][4]Enter coefficient for 0x5613b3702320[4][5]Enter coefficient for 0x5613b3702320[4][6]Enter coefficient for 0x5613b3702320[4][7]Enter coefficient for 0x5613b3702320[5][0]Enter coefficient for 0x5613b3702320[5][1]Enter coefficient for 0x5613b3702320[5][2]Enter coefficient for 0x5613b3702320[5][3]Enter coefficient for 0x5613b3702320[5][4]Enter coefficient for 0x5613b3702320[5][5]Enter coefficient for 0x5613b3702320[5][6]Enter coefficient for 0x5613b3702320[5][7]Enter coefficient for 0x5613b3702320[6][0]Enter coefficient for 0x5613b3702320[6][1]Enter coefficient for 0x5613b3702320[6][2]Enter coefficient for 0x5613b3702320[6][3]Enter coefficient for 0x5613b3702320[6][4]Enter coefficient for 0x5613b3702320[6][5]Enter coefficient for 0x5613b3702320[6][6]Enter coefficient for 0x5613b3702320[6][7]Enter coefficient for 0x5613b3702320[7][0]Enter coefficient for 0x5613b3702320[7][1]Enter coefficient for 0x5613b3702320[7][2]Enter coefficient for 0x5613b3702320[7][3]Enter coefficient for 0x5613b3702320[7][4]Enter coefficient for 0x5613b3702320[7][5]Enter coefficient for 0x5613b3702320[7][6]Enter coefficient for 0x5613b3702320[7][7]
 ===== CONVERGENCE PREDICTION: may not conv =====
PRINTING VECTOR FROM: 0x5613b37022f0 LOCATION :
1.000000 1.000000 1.000000 0.999999 0.999999 0.999999 0.999999 1.000000
EPSILON CALCULATED 58
\end{lstlisting}
\begin{lstlisting}[language=C,inputencoding=utf8, basicstyle=\fontsize{8}{10}\selectfont,caption=Execution with A6 dimensions: 10x10]
A matrix dimensions: 
Enter coefficient for 0x5634c724b320[0][0]Enter coefficient for 0x5634c724b320[0][1]Enter coefficient for 0x5634c724b320[0][2]Enter coefficient for 0x5634c724b320[0][3]Enter coefficient for 0x5634c724b320[0][4]Enter coefficient for 0x5634c724b320[0][5]Enter coefficient for 0x5634c724b320[0][6]Enter coefficient for 0x5634c724b320[0][7]Enter coefficient for 0x5634c724b320[0][8]Enter coefficient for 0x5634c724b320[0][9]Enter coefficient for 0x5634c724b320[1][0]Enter coefficient for 0x5634c724b320[1][1]Enter coefficient for 0x5634c724b320[1][2]Enter coefficient for 0x5634c724b320[1][3]Enter coefficient for 0x5634c724b320[1][4]Enter coefficient for 0x5634c724b320[1][5]Enter coefficient for 0x5634c724b320[1][6]Enter coefficient for 0x5634c724b320[1][7]Enter coefficient for 0x5634c724b320[1][8]Enter coefficient for 0x5634c724b320[1][9]Enter coefficient for 0x5634c724b320[2][0]Enter coefficient for 0x5634c724b320[2][1]Enter coefficient for 0x5634c724b320[2][2]Enter coefficient for 0x5634c724b320[2][3]Enter coefficient for 0x5634c724b320[2][4]Enter coefficient for 0x5634c724b320[2][5]Enter coefficient for 0x5634c724b320[2][6]Enter coefficient for 0x5634c724b320[2][7]Enter coefficient for 0x5634c724b320[2][8]Enter coefficient for 0x5634c724b320[2][9]Enter coefficient for 0x5634c724b320[3][0]Enter coefficient for 0x5634c724b320[3][1]Enter coefficient for 0x5634c724b320[3][2]Enter coefficient for 0x5634c724b320[3][3]Enter coefficient for 0x5634c724b320[3][4]Enter coefficient for 0x5634c724b320[3][5]Enter coefficient for 0x5634c724b320[3][6]Enter coefficient for 0x5634c724b320[3][7]Enter coefficient for 0x5634c724b320[3][8]Enter coefficient for 0x5634c724b320[3][9]Enter coefficient for 0x5634c724b320[4][0]Enter coefficient for 0x5634c724b320[4][1]Enter coefficient for 0x5634c724b320[4][2]Enter coefficient for 0x5634c724b320[4][3]Enter coefficient for 0x5634c724b320[4][4]Enter coefficient for 0x5634c724b320[4][5]Enter coefficient for 0x5634c724b320[4][6]Enter coefficient for 0x5634c724b320[4][7]Enter coefficient for 0x5634c724b320[4][8]Enter coefficient for 0x5634c724b320[4][9]Enter coefficient for 0x5634c724b320[5][0]Enter coefficient for 0x5634c724b320[5][1]Enter coefficient for 0x5634c724b320[5][2]Enter coefficient for 0x5634c724b320[5][3]Enter coefficient for 0x5634c724b320[5][4]Enter coefficient for 0x5634c724b320[5][5]Enter coefficient for 0x5634c724b320[5][6]Enter coefficient for 0x5634c724b320[5][7]Enter coefficient for 0x5634c724b320[5][8]Enter coefficient for 0x5634c724b320[5][9]Enter coefficient for 0x5634c724b320[6][0]Enter coefficient for 0x5634c724b320[6][1]Enter coefficient for 0x5634c724b320[6][2]Enter coefficient for 0x5634c724b320[6][3]Enter coefficient for 0x5634c724b320[6][4]Enter coefficient for 0x5634c724b320[6][5]Enter coefficient for 0x5634c724b320[6][6]Enter coefficient for 0x5634c724b320[6][7]Enter coefficient for 0x5634c724b320[6][8]Enter coefficient for 0x5634c724b320[6][9]Enter coefficient for 0x5634c724b320[7][0]Enter coefficient for 0x5634c724b320[7][1]Enter coefficient for 0x5634c724b320[7][2]Enter coefficient for 0x5634c724b320[7][3]Enter coefficient for 0x5634c724b320[7][4]Enter coefficient for 0x5634c724b320[7][5]Enter coefficient for 0x5634c724b320[7][6]Enter coefficient for 0x5634c724b320[7][7]Enter coefficient for 0x5634c724b320[7][8]Enter coefficient for 0x5634c724b320[7][9]Enter coefficient for 0x5634c724b320[8][0]Enter coefficient for 0x5634c724b320[8][1]Enter coefficient for 0x5634c724b320[8][2]Enter coefficient for 0x5634c724b320[8][3]Enter coefficient for 0x5634c724b320[8][4]Enter coefficient for 0x5634c724b320[8][5]Enter coefficient for 0x5634c724b320[8][6]Enter coefficient for 0x5634c724b320[8][7]Enter coefficient for 0x5634c724b320[8][8]Enter coefficient for 0x5634c724b320[8][9]Enter coefficient for 0x5634c724b320[9][0]Enter coefficient for 0x5634c724b320[9][1]Enter coefficient for 0x5634c724b320[9][2]Enter coefficient for 0x5634c724b320[9][3]Enter coefficient for 0x5634c724b320[9][4]Enter coefficient for 0x5634c724b320[9][5]Enter coefficient for 0x5634c724b320[9][6]Enter coefficient for 0x5634c724b320[9][7]Enter coefficient for 0x5634c724b320[9][8]Enter coefficient for 0x5634c724b320[9][9]
 ===== CONVERGENCE PREDICTION: may not conv =====
PRINTING VECTOR FROM: 0x5634c724b2f0 LOCATION :
1.000000 1.000000 1.000000 1.000000 1.000000 0.999999 0.999999 0.999999 0.999999 1.000000
EPSILON CALCULATED 70
\end{lstlisting}
\subsection{Remarque sur les résultats}
On remarquera que lorsque Jacobi renvoie \textbf{NaN}, Il s'agit du cas où cette méthode diverge, donc aucun résultat fiable n'est envisageable.\\
On rappelera aussi que dans le cadre de ce programme, si la méthode renvoie des valeurs \textbf{extremêment proches} de 1, alors le programme a trouvé une solution. 
\end{document}                                                                                                                                                                                                                                              
\documentclass{report}
\setlength{\headheight}{24.1638pt}
%packages
\usepackage[french]{babel}
\usepackage[T1]{fontenc}
\usepackage[utf8]{inputenc}
\usepackage{mathtools}
\usepackage{amssymb}
\usepackage{hyperref}
\usepackage{float}
\usepackage{amsthm}
\usepackage{listings}
\usepackage{geometry}
\usepackage{setspace}
\usepackage{graphicx}
\usepackage{fancyhdr}
\usepackage{subcaption}
\usepackage{cleveref}

%commands
\newtheorem{defi}{Définition}
\renewcommand{\thedefi}{\empty{}}

\renewcommand\headrulewidth{1pt}
\newcommand{\crule}[3][c]{%
    \par\noindent
    \makebox[\linewidth][#1]{\rule{#2\linewidth}{#3}}}
\renewcommand{\thechapter}{\Roman{chapter}}

%Style de page
\pagestyle{fancy}
\fancyhead[L]{}
\fancyhead[C]{}
\fancyhead[R]{\leftmark}
\allowdisplaybreaks
\geometry{hmargin=3cm,vmargin=2.5cm}

%préambule
\begin{document}
\section{Conclusion Générale des Méthodes Itératives}
\subsection{Tableau récapitulatif}
$
\begin{array}{|c|c|c|c|c|}
\hline
A_i & p(J) & \text{NbIterJacobi} & p(GS) & \text{NbIterGauss-Seidel} \\
\hline
A_1 & \infty & 742 & \infty & 192\\
\hline
A_2 & \infty & 252 & \infty & 128 \\
\hline
A_3 & 0.000001& 19 & 0.00000 & 9 \\
\hline
A_4 & 0.000001 & 35 & 0.000001& 59 \\
\hline
A_5 & 0.000001 & 25 & 0.000001& 14 \\
\hline
A_6 & 0.000002 & 26 & 0.000001 & 15 \\
\hline
A_7 & 0.000001 & 26 & 0.000001 & 15 \\
\hline
A_8 & 0.000002 & 26 & 0.000001 & 15 \\
\hline
A_9 & 0.000002 & 35 & 0.000001 & 19 \\ 
\hline
A_{10} & 0.000002 & 88 & 0.000001 & 44 \\
\hline
A_{11} & 0.000001 & 120 & 0.000001& 60 \\
\hline
A_{12} & 0.000001 & 140 & 0.000001 & 73 \\
\hline
\end{array}
$
\subsection{Conclusion}
Comme peuvent le démontrer les différents graphiques ainsi que le tableau ci-dessus, nous remarquerons que la Méthode de \textbf{Gauss-Seidel} reste majoritairement plus efficace que la méthode de \textbf{Jacobi}. \\
Nous insisterons sur le fait que la Méthode de \textbf{Gauss-Seidel} est particulièrement adaptée pour le calcul parallèle alors que la méthode de \textbf{Jacobi} est plus adaptée sur des matrices creuses.
\end{document}
\end{array}
\chapter*{Annexe}
\addcontentsline{toc}{chapter}{Annexe}
\label{annexe}
\section*{Matrices Test}
\addcontentsline{toc}{section}{Matrices Test}
$A_1 = \begin{pmatrix}
    3&0&4\\
    7&4&2\\
    -1&1&2\\
    \end{pmatrix}
    $
    ,
    $A_2 = \begin{pmatrix}
    -3&3&-6\\
    -4&7&8\\
    5&7&-9\\
    \end{pmatrix}
    $,
    $A_3 = \begin{pmatrix}
    4&1&1\\
    2&-9&0\\
    0&-8&6\\
    \end{pmatrix}
    $,
    $A_4 = \begin{pmatrix}
    7&6&9\\
    4&5&-4\\
    -7&-3&8\\
    \end{pmatrix}
    $,\vspace{6pt}\\
    \begin{equation*}
        A_5=
        \begin{cases}
            a_{i,i}=1 \\
            a_{1,j}=a_{j,1}=2^{1-j}\\
            0 \text{ sinon}
        \end{cases} \text{pour }i,j=1,....,3
    \end{equation*}
    ,\\
    \begin{equation*}
        A_6=
        \begin{cases}
            a_{i,i}=1 \\
            a_{1,j}=a_{j,1}=2^{1-j}\\
            0 \text{ sinon}
        \end{cases} \text{pour }i,j=1,....,6
    \end{equation*}
    ,\\
    \begin{equation*}
        A_7=
        \begin{cases}
            a_{i,i}=1 \\
            a_{1,j}=a_{j,1}=2^{1-j}\\
            0 \text{ sinon}
        \end{cases} \text{pour }i,j=1,....,8
    \end{equation*}
    ,\\
    \begin{equation*}
        A_8=
        \begin{cases}
            a_{i,i}=1 \\
            a_{1,j}=a_{j,1}=2^{1-j}\\
            0 \text{ sinon}
        \end{cases} \text{pour }i,j=1,....,10
    \end{equation*}
    ,\\
    \begin{equation*}
        A_9=
        \begin{cases}
            a_{i,i}=3 \\
            a_{i,j}=-1 \text{ si } j=i+1, i<n\\
            a_{i,j}=-2 \text{ si } j=i-1, i>1 \\
            0 \text{ sinon}
        \end{cases} \text{pour }i,j=1,....,3
    \end{equation*}
    ,\\
    \begin{equation*}
        A_{10}=
        \begin{cases}
            a_{i,i}=3 \\
            a_{i,j}=-1 \text{ si } j=i+1, i<n\\
            a_{i,j}=-2 \text{ si } j=i-1, i>1 \\
            0 \text{ sinon}
        \end{cases} \text{pour }i,j=1,....,6
    \end{equation*}
    ,\\
    \begin{equation*}
        A_{11}=
        \begin{cases}
            a_{i,i}=3 \\
            a_{i,j}=-1 \text{ si } j=i+1, i<n\\
            a_{i,j}=-2 \text{ si } j=i-1, i>1 \\
            0 \text{ sinon}
        \end{cases} \text{pour }i,j=1,....,8
    \end{equation*}
    ,\\
    \begin{equation*}
        A_{12}=
        \begin{cases}
            a_{i,i}=3 \\
            a_{i,j}=-1 \text{ si } j=i+1, i<n\\
            a_{i,j}=-2 \text{ si } j=i-1, i>1 \\
            0 \text{ sinon}
        \end{cases} \text{pour }i,j=1,....,10
    \end{equation*}
\end{document}                                                                          