\chapter{Résolution de systèmes linéaires par des Méthodes Directes: Méthode de Gauss}
Dans le cadre de ce premier TP, nous devions implémenter l'algorithme du \emph{Pivot de Gauss} en utilisant le langage de programmation C.\\
\textit{Afin de rendre ce document plus compréhensible et lisible, nous estimons que la présence de nos codes sources en clair est nécessaire.} \\
\section{Détail de l'algorithme}
Soient deux matrices $A \in \mathcal{M}_{m,m} \text{ et  } b \in \mathcal{M}_{m,1}$ . \\
L'algorithme de Gauss se décrit ainsi: \\
\label{algo}
\begin{lstlisting}[mathescape=true, frame=single]
Pour $k = 1,\ldots,n-1$ Faire:
	Pour $i=k+1,\ldots,n$ Faire:

				$\alpha_i^{(k)} = \frac{a_{ik}^{(k)}}{a_{kk}^{(k)}}$ 
		
		Pour $j=k,\ldots,n$ Faire:

		   		$a_{ij}^{(k+1)} = a_{ij}^{(k)} - \alpha_i^{(k)}a_{kj}^{(k)}$
	
		FIN Pour $j$

				$b_i^{(k+1)} = b_i^{(k)} - \alpha_i^{(k)}b_k^{(k)}$

	FIN Pour $i$
FIN Pour $k$
\end{lstlisting}
\newpage
Une fois la matrice échelonnée par cet algorithme, on appliquera la formule suivante pour trouver les solutions du système : \\
\begin{mdframed}
\begin{center}
\begin{large}
$ x_n = \frac{b_n}{a_{m,m}}$ \\
\end{large}
\end{center}
et \\
\begin{center}
\begin{large}
$ \forall i = n-1, \ldots, 1$, $x_i = \frac{1}{a_{ii}}\left( b_i-\sum\limits_{j=1+i}^n a_{ij}x_j \right)$\\
\end{large}
\end{center}
\end{mdframed}
La complexité temporelle de cet algorithme est cubique soit $O(n^3)$ avec une complexité exacte de $\frac{2n^3}{3}$. \\
Pour l'implémentation de cet algorithme, nous allons présenter deux façons de le conceptualiser avec une comparaison algorithmique des deux programmes. \\
\section{Le pivot de Gauss en pratique}
\subsection{De manière générale}
Soit $A \in \mathcal{M}_{m,m}$ et $B \in \mathcal{M}_{m,1}$ et $x$ la matrice des inconnues.\\
Considérons alors le système suivant $Ax=b$.\\
Ce système peut être représenté sous la forme d'une matrice augmentée $M$ tel que: \vspace{10pt}\\
$
M =  \begin{pmatrix}
a_{11}& \ldots & a_{1m} &\bigm| & b_{1} \\
\vdots & \ddots & \vdots&\bigm| &  \vdots \\
a_{m1} & \ldots & a_{mm} &\bigm|& b_{m} 
\end{pmatrix}
$
\vspace{10pt}\\
Après exécution du pivot de Gauss, $M$ devient \vspace{10pt}\\
$
M =  \begin{pmatrix}
1& a_{12}& a_{13} &\ldots & a_{1m} &\bigm| & b_{1}^{'} \\
0 & 1 & a_{23}^{'}&  \ldots &a_{2m}^{'} &\bigm| & b_{2}^{'} \\
0 & 0 & 1 & \ldots &\vdots &\bigm| & b_{3}^{'} \\
\vdots & \vdots & \vdots& \ddots& \vdots &\bigm| &  \vdots \\
0 & \ldots &0 &\ldots &1&\bigm|& b_{m}^{'}
\end{pmatrix}
$  \vspace{10pt}\\
Une fois que tous les pivots sont placés, il suffira de reconstituer le système et de le remonter afin de déterminer les inconnues comme suit:\vspace{10pt}\\
Soit $A^{'}=  \begin{pmatrix}
1& a_{12}^{'}& a_{13}^{'} &\ldots & a_{1m}^{,} \\
0 & 1 & a_{23}^{'}&  \ldots &a_{2m}^{'} \\
0 & 0 & 1 & \ldots &\vdots\\
\vdots & \vdots & \vdots& \ddots& \vdots \\
0 & \ldots &0 &\ldots &1
\end{pmatrix}
$
et $b = \begin{pmatrix}
b_{1}^{'} \\
b_{2}^{'} \\
b_{3}^{'} \\
\vdots \\
 b_{m}^{'}
\end{pmatrix}
$ \\
alors pour $A^{'}x = b$ on a donc \begin{large}$ x_n = \frac{b_n}{a_{m,m}}$\end{large} et \begin{large}$x_i = \frac{1}{a_{ii}}\left( b_i-\sum\limits_{j=1+i}^n a_{ij}x_j \right), \forall i = n-1, \ldots, 1$\end{large}. \vspace{8pt}\\
\textit{Nous remarquerons que l'implémentation du pivot de Gauss ne nécessitera pas de mettre nos pivots à $1$}

\subsection{Exercice}
\begin{mdframed}
Résoudre le système linéaire suivant : 
\begin{System}
  x + y + 2z = 3 \\
  x + 2y + z = 1 \\
  2x + y + z = 0 
\end{System}
\end{mdframed}
On pose $A= 
\begin{pmatrix}
1 & 1 & 2 \\
1 & 2 & 1 \\
2 & 1 & 1 \\
\end{pmatrix}$, 
$B= 
\begin{pmatrix}
3 \\
1 \\
0 \\
\end{pmatrix}$ 
et
$X= 
\begin{pmatrix}
x \\
y \\
z \\
\end{pmatrix}$ \vspace{6pt}\\
puis la matrice augmentée 
$\begin{pmatrix}
    A & \big| & B \\
\end{pmatrix}
=
\begin{pmatrix}
    1 & 1 & 2 & \big| & 3 \\
    1 & 2 & 1 & \big| & 1 \\
    2 & 1 & 1 & \big| & 0 \\
\end{pmatrix}$  \vspace{8pt}\\
Commençons par échelonner la matrice augmentée à l'aide du pivot de Gauss: \vspace{8pt}\\
$
\begin{pmatrix}
    1 & 1 & 2 & \big| & 3 \\
    1 & 2 & 1 & \big| & 1 \\
    2 & 1 & 1 & \big| & 0 \\
\end{pmatrix}
\xrightarrow[L_2-L_1 \rightarrow L_2]{-(L_3+2L_1) \rightarrow L_3}
\begin{pmatrix}
    1 & 1 & 2 & \big| & 3 \\
    0 & 1 & -1 & \big| & -2 \\
    0 & 1 & 3 & \big| & 6 \\
\end{pmatrix}
\xrightarrow{\frac{L_3-L_2}{4} \rightarrow L_3}
\begin{pmatrix}
    1 & 1 & 2 & \big| & 3 \\
    0 & 1 & -1 & \big| & -2 \\
    0 & 0 & 1 & \big| & 2 \\
\end{pmatrix}
$ \vspace{8pt}\\
Maintenant que notre matrice augmentée est échelonnée, nous pouvons déterminer les inconnues du système par substitution (en partant du bas): \vspace{8pt}\\
$
\begin{pmatrix}
    1 & 1 & 2 & \big| & 3 \\
    0 & 1 & -1 & \big| & -2 \\
    0 & 0 & 1 & \big| & 2 \\
\end{pmatrix}
\xrightarrow[L_1-2L_3 \rightarrow L_1]{L_2+L_3 \rightarrow L_2}
\begin{pmatrix}
    1 & 1 & 0 & \big| & -1 \\
    0 & 1 & 0 & \big| & 0 \\
    0 & 0 & 1 & \big| & 2 \\
\end{pmatrix}
\xrightarrow{L_1-L_2 \rightarrow L_1}
\begin{pmatrix}
    1 & 0 & 0 & \big| & -1 \\
    0 & 1 & 0 & \big| & 0 \\
    0 & 0 & 1 & \big| & 2 \\
\end{pmatrix}
$ \vspace{8pt}\\

Nous avons maintenant $A=I_3$ et donc nous pouvons remplacer dans le système $AX=B$, $A$ par $I_3$, ce qui nous donne: \vspace{4pt}\\
$
AX=B \iff I_3X=B
\iff
    \begin{pmatrix}
        1 & 0 & 0\\
        0 & 1 & 0 \\
        0 & 0 & 1
    \end{pmatrix}
    \begin{pmatrix}
    	x \\
		y \\
		z \\
    \end{pmatrix}
    =
    \begin{pmatrix}
    	-1\\
		0 \\
		2 \\
    \end{pmatrix}
    \iff
    \begin{pmatrix}
    	x \\
		y \\
		z \\
    \end{pmatrix}
    =
    \begin{pmatrix}
    	-1\\
		0 \\
		2 \\
    \end{pmatrix}$
\vspace{6pt}\\
Nous avons alors notre couple solution du système, qui est : \vspace{4pt}\\
\begin{System}
  x = -1 \\
  y = 0 \\
  z = 2 
\end{System}\\

\newpage
\lstset{
  firstnumber=0, 
  numbers=left,               
  frame=single,
  language=C,                                       
  showstringspaces=false
}
\section{Implémentation de l'algorithme de Gauss en passant par le système d'équations linéaires}
\subsection{Code source}
Voici mon implémentation de l'algorithme de Gauss, qui n'utilise pas la matrice augmentée. En effet, l'algorithme travaille directement avec le système d'équations linéaires Ax=B.\\
\begin{lstlisting}[language=C,inputencoding=utf8, basicstyle=\fontsize{8}{10}\selectfont]
#include <stdio.h>
#include <string.h>
#include <stdlib.h>

/*
*CREATE A 2D FLOAT MATRIX
*/

float** createMatrix(int row, int column){
	float **mat=NULL;
	mat=malloc(row* sizeof(int*));
	if(mat==NULL){return NULL;}
	for (int i=0; i<row; i++){
		mat[i]=malloc(column* sizeof(int));
		if(mat[i]==NULL){
			for(int j=0; j<i; j++){
				free(mat[j]);
				return NULL;
			}
		}
	}
	return mat;
}

/*
*PRINT A 2D FLOAT MATRIX
*/

void printMatrix(float **mat, int row, int column){
	for (int i=0; i<row; i++){
    		for(int j=0; j<column; j++){
         		printf("%f   ", mat[i][j]);
    		}
    		printf("\n");
	}
}

/*
*FREE A 2D FLOAT MATRIX
*/

void freeMatrix(float **mat, int row){
	for(int i=0; i<row;i++){
		free(mat[i]);
	}
	free(mat);
}

/*
*COMPLETE A 2D FLOAT MATRIX FROM USER INPUT
*/

void completeMatrix(float **mat, int row, int column){
	for (int i=0; i<row; i++){
    		for(int j=0; j<column; j++){
    			printf("Coefficient at M_%d,%d:   ", i+1, j+1);
         		scanf("%f", &mat[i][j]);
    		}
	}
}

/*
*GENERATE A COLUMN VECTOR "B" FROM A 2D FLOAT MATRIX "A"
*/

void generateB(float **matA, float **matB, int row, int column){
	for(int i=0; i<row; i++){
		float sum=0;
		for(int j=0; j<column; j++){
			sum+=matA[i][j];
		}
		matB[i][0]=sum;
    	}
}

/*
*PERFORM GAUSSIAN ELIMINATION ON A Ax=B MATRIX SYSTEM OF LINEAR EQUATIONS
*/

void gauss(float** matA, float** matb, int size){
	for(int k=0; k<size-1; k++){
		for(int i=k+1; i<size; i++){
			float alpha=matA[i][k]/matA[k][k];
			for(int j=k; j<size; j++){
				matA[i][j]=matA[i][j]-alpha*matA[k][j];
			}
			matb[i][0]=matb[i][0]-alpha*matb[k][0];
		}
	}
}

/*
*SOLVE A MATRIX SYSTEM OF LINEAR EQUATIONS USING BACKWARD SUBSTITUTION
*/
void resolution(float** matA, float** matb, float** matx, int size){
	matx[size-1][0]=matb[size-1][0]/matA[size-1][size-1];
	for (int i=size-2; i>=0; i--){
		float sum=0;
		for(int j=i+1; j<size; j++){
			sum+=matA[i][j]*matx[j][0];
		}
		matx[i][0]=(1/matA[i][i])*(matb[i][0]-sum);
	}
}

int main(){

	//A Matrix
	int rowA;
	int columnA;
	printf("\nRow count of matrix A : ");
	scanf("%d", &rowA);
	printf("\nColumn count of matrix A : ");
	scanf("%d", &columnA);

	float** Amatrix=createMatrix(rowA, columnA);
	completeMatrix(Amatrix, rowA, columnA);
	puts("\n		A matrix \n");
	printMatrix(Amatrix, rowA, columnA);
	
	//B Matrix
	float** Bmatrix=createMatrix(rowA, 1);
	generateB(Amatrix, Bmatrix, rowA, columnA);
	puts("\n		B matrix \n");
	printMatrix(Bmatrix, rowA, 1);
	
	//X Matrix
	float** Xmatrix=createMatrix(rowA, 1);
	
	//Matrix Triangularization
	puts("\n		TRIANGULARIZATION \n");
	gauss(Amatrix, Bmatrix, rowA);
	puts("\n		A Matrix \n");
	printMatrix(Amatrix, rowA, columnA);
	puts("\n		B Matrix \n");
	printMatrix(Bmatrix, rowA, 1);
	
	//Solve the system
	puts("\n		SOLVING \n");
	resolution(Amatrix, Bmatrix, Xmatrix, rowA);
	puts("\n		SOLUTION VECTOR X \n");
	printMatrix(Xmatrix, rowA, 1);
	
	//Free
	freeMatrix(Amatrix, rowA);
	freeMatrix(Bmatrix, rowA);
	freeMatrix(Xmatrix, rowA);
	return 0;
}
\end{lstlisting}
\subsection{Commentaires}
\subsubsection{Fonctions usuelles de manipulation de matrices}\label{fonctusu}
Ce code implémente diverses fonctions pour travailler avec des matrices à coefficients en nombre flottants.\\
\begin{itemize}
\item La fonction \textit{\textbf{createMatrix}} alloue dynamiquement de la mémoire pour créer une matrice de nombres flottants avec un nombre spécifié de lignes et de colonnes.
\item La fonction \textit{\textbf{printMatrix}} affiche les éléments d'une matrice de nombres flottants.
\item La fonction \textit{\textbf{freeMatrix}} libère la mémoire allouée pour une matrice de nombres flottants.
\item La fonction \textit{\textbf{completeMatrix}} permet à l'utilisateur de saisir des valeurs pour remplir les éléments d'une matrice de nombres flottants.
\item La fonction \textit{\textbf{generateB}} génère un vecteur colonne $B$ en fonction de la somme des éléments de chaque ligne de la matrice $A$.
\end{itemize}

\subsubsection{Fonctions résolvant notre système linéaire $Ax=B$ à l'aide de l'algorithme de Gauss}
Dans le cadre de notre résolution de systèmes d'équations linéaires, deux fonctions jouent des rôles clefs dans ce code : la fonction \textit{\textbf{gauss}} et la fonction \textit{\textbf{resolution}}.\\

\begin{itemize}
\item La fonction \textit{\textbf{gauss}} joue un rôle important dans la préparation de la résolution de notre système d'équations linéaires. En effectuant l'élimination de Gauss sur la matrice $A$, elle la transforme en une matrice triangulaire supérieure. Cela signifie que les éléments sous la diagonale principale de la matrice deviennent tous des zéros, simplifiant ainsi la résolution du système. De plus, la fonction met également à jour la matrice $B$ en conséquence, garantissant que notre système $Ax=B$ reste équilibré.\\

\item La fonction \textit{\textbf{resolution}}, quant à elle, prend en charge la résolution effective du système linéaire une fois que la matrice $A$ a été triangulée par la fonction \textbf{gauss}. Elle utilise la méthode de substitution pour calculer la solution et stocke le résultat dans le vecteur $X$. Cette étape finale permet d'obtenir les valeurs des variables inconnues du système, fournissant ainsi la solution recherchée pour le problème initial.\\
\end{itemize}


En combinant ces deux fonctions avec celles citées dans la sous-section \ref{fonctusu}, le code réalise un processus complet de résolution de systèmes d'équations linéaires de manière efficace et précise (aux erreurs d'arrondies près).\\
\subsection{Interaction Utilisateur/Console}
\subsubsection{Entrées utilisateur}

En premier lieu dans notre programme, nous avons besoin de spécifier le système $Ax=B$ à l'ordinateur. 
Pour ce faire, nous allons dans l'ordre :
\begin{enumerate}
\item Allouer une matrice $A$ en mémoire. Cette matrice verra sa taille définie par la première entrée utilisateur du programme (nous demanderons consécutivement le nombre de lignes, puis le nombre de colonnes de la matrice).
\item Définir les coefficients de la matrice $A$. Il s'agira de la deuxième entrée utilisateur de notre programme. \\
Par définition de notre fonction \textit{\textbf{completeMatrix}}, nous remplirons la matrice dans l'ordre suivant:\\
$a_{1,1}, a_{1,2}, ..., a_{1,n}, \text{   puis   } a_{2,1}, ... a_{2,n}, \text{   jusque   }  a_{n,1}, ..., a_{n,n}$
\item Allouer une matrice $B$ en mémoire. À noter que la taille de $B$ est définie automatiquement en fonction de la taille de $A$. Nous avons $A\in \mathcal{M}_{n,p} \Rightarrow B\in \mathcal{M}_{n,1}$.
\item Définir les coefficients de la matrice $B$. Chaque coefficient prendra la valeur de la somme des éléments de la ligne respective de la matrice $A$.\\
Nous avons donc:\\ $ \text{Soient } A\in \mathcal{M}_{n,p} \text{ et } B\in \mathcal{M}_{n,1}  , \forall i \in \{1,n\}  , b_{i,1}=\sum_{j=1}^{p} a_{i,p}$.\\
\item Allouer une matrice $X$ en mémoire. Cette matrice aura la même taille que la matrice $B$. Ces coefficients ne seront pas définis pour le moment.
\end{enumerate}

En guise d'exemple, le système matriciel $AX=B$ suivant:

\begin{equation}
\begin{pmatrix}
3 & 0 & 4\\
7 & 4 & 2 \\
-1 & 1 & 2
\end{pmatrix} 
\begin{pmatrix}
x_1\\
x_2\\
x_3\\
\end{pmatrix}
=
\begin{pmatrix}
7 \\
13 \\
2
\end{pmatrix}
\end{equation}
\\


est représenté par l'entrée utilisateur:
\begin{lstlisting}[caption=User Input, basicstyle=\fontsize{6}{8}\selectfont]
Row count of matrix A : 3

Column count of matrix A : 3

		FILL IN THE VALUE OF MATRIX A 

Value for a_1,1:   3
Value for a_1,2:   0
Value for a_1,3:   4
Value for a_2,1:   7
Value for a_2,2:   4
Value for a_2,3:   2
Value for a_3,1:   -1
Value for a_3,2:   1
Value for a_3,3:   2

\end{lstlisting}
Une fois toutes les matrices initialisées et complétées, nous pouvons attaquer la résolution du système par la triangularisation du système. Ceci fait, nous résolverons le système obtenu pour obtenir notre vecteur $X$ solution.\\
\subsubsection{Affichage Console}
Dès lors le système $AX=B$ connu par l'ordinateur, ce dernier peut retrouver les valeurs de la matrice $X$. Voici l'affichage produit par notre programme en console: \\
\begin{lstlisting}[caption=Console Display of the Gauss elimination for the AX=B system mentioned above, basicstyle=\fontsize{6}{8}\selectfont]
		A matrix 

3.000000   0.000000   4.000000   
7.000000   4.000000   2.000000   
-1.000000   1.000000   2.000000   

		B matrix 

7.000000   
13.000000   
2.000000   

		TRIANGULARIZATION 
		A Matrix 

3.000000   0.000000   4.000000   
0.000000   4.000000   -7.333333   
0.000000   0.000000   5.166667   

		B Matrix 

7.000000   
-3.333332   
5.166667   

		SOLVING 
		SOLUTION VECTOR X 

1.000000   
1.000000   
1.000000   

\end{lstlisting}
Il est à repérer que le programme affiche dans cet ordre: \\
\begin{itemize}
\item La Matrice $A$ 
\item La Matrice $B$
\item La Matrice $A$ une fois triangulée supérieure
\item La Matrice $B$ une fois mise à jour en conséquence pour que le système reste équilibré
\item La Matrice $X$ solution du système
\end{itemize}

\textit{\underline{Remarque}: le temps d'exécution de ce programme a été de 0.000237 secondes}
\subsection{Exemples d'exécutions}

Soient les matrices suivantes données dans le TP:\\

$A_2 = \begin{pmatrix}
-3 & 3 & -6 \\
-4 & 7 &  8 \\
5 & 7 & -9 \\
\end{pmatrix}
$
,
$A_4 = \begin{pmatrix}
7 & 6 & 9 \\
4 & 5 &  -4\\
-7 & -3 & 8 \\
\end{pmatrix}
$,
$A_6 = \begin{pmatrix}
-3 & 3 & -6 \\
-4 & 7 &  8 \\
5 & 7 & -9 \\
\end{pmatrix}
$
\vspace{12pt}\\
On obtient respectivement les résultats suivants:
\\
\begin{lstlisting}[caption={$A_2X=B$} results, basicstyle=\fontsize{4}{6}\selectfont]
  		A matrix 

-3.000000   3.000000   -6.000000   
-4.000000   7.000000   8.000000   
5.000000   7.000000   -9.000000   

		B matrix 

-6.000000   
11.000000   
3.000000   

		TRIANGULARIZATION
		A Matrix 

-3.000000   3.000000   -6.000000   
0.000000   3.000000   16.000000   
0.000000   0.000000   -83.000000   

		B Matrix 

-6.000000   
19.000000   
-83.000000   

		SOLVING 
		SOLUTION VECTOR X 

1.000000   
1.000000   
1.000000   

Temps d'execution : 0.000250 secondes
\end{lstlisting}
\begin{lstlisting}[caption={$A_4X=B$} results, basicstyle=\fontsize{4}{6}\selectfont]
		A matrix 

7.000000   6.000000   9.000000   
4.000000   5.000000   -4.000000   
-7.000000   -3.000000   8.000000   

		B matrix 

22.000000   
5.000000   
-2.000000   

		TRIANGULARIZATION 
		A Matrix 

7.000000   6.000000   9.000000   
0.000000   1.571428   -9.142858   
0.000000   0.000000   34.454552   

		B Matrix 

22.000000   
-7.571429   
34.454548   

		SOLVING 
		SOLUTION VECTOR X 

1.000001   
0.999999   
1.000000   

Temps d'execution : 0.000231 secondes
\end{lstlisting}
\begin{lstlisting}[caption={$A_6X=B$} results, basicstyle=\fontsize{4}{6}\selectfont]
		A matrix 

-3.000000   3.000000   -6.000000   
-4.000000   7.000000   8.000000   
5.000000   7.000000   -9.000000   

		B matrix 

-6.000000   
11.000000   
3.000000   

		TRIANGULARIZATION 
		A Matrix 

-3.000000   3.000000   -6.000000   
0.000000   3.000000   16.000000   
0.000000   0.000000   -83.000000   

		B Matrix 

-6.000000   
19.000000   
-83.000000   

		SOLVING 
		SOLUTION VECTOR X 

1.000000   
1.000000   
1.000000   

Temps d'execution : 0.000246 secondes
\end{lstlisting}

\textbf{On remarquera} que sur le calcul de $A_4$, on tombe sur des valeur extrêmement proche de $1$. Ceci est provoqué à cause des erreurs d'arrondis provoqués par l'encodage des nombres flottants.     
\newpage
\documentclass{report}
\usepackage[T1]{fontenc}
\usepackage[utf8]{inputenc}
\usepackage{mathtools}
\usepackage{amssymb}
\usepackage{hyperref}
\usepackage{float}
\usepackage{amsthm}
\usepackage{listings}
\usepackage{geometry}
\usepackage{setspace}
\usepackage{graphicx}
\usepackage{fancyhdr}
\usepackage{subcaption}
\usepackage{cleveref}

\begin{document}
\subsection{Implémentation grâce à une matrice augmentée}
\subsubsection{Code source}
Voici le code source de mon implémentation du pivot de Gauss via le passage par la matrice augmentée.\\
C'est-à-dire que dans mon implémentation il y a une concaténation des matrices.
\begin{lstlisting}[language=C]
#include <stdio.h>
#include <stdlib.h>
#include <string.h>
#include <time.h>
/*
 * PRINT MATRIX WITH RIGHT FORMAT
 */
void printMatrix(float **matrix, int m, int p) {
  printf("PRINTING MATRIX FROM: %p LOCATION :\n", matrix);
  for (int i = 0; i < m; i++) {
    for (int j = 0; j < p; j++) {
      (j <= p - 2) ? printf("%f ", matrix[i][j]) : printf("%f", matrix[i][j]);
    }
    puts("");
  }
}
/*
 * ALLOCATE MEMORY FOR MATRIX
 */
float **allocate(int m, int n) {
  float **T = malloc(m * sizeof *T);
  for (int i = 0; i < m; i++) {
    T[i] = malloc(n * sizeof *T[i]);
    if (T[i] == NULL) {
      for (int j = 0; j < i; j++) {
        free(T[i]);
      }
      free(T);
      puts("ALLOCATION ERROR");
      exit(-1);
    }
  }
  return T;
}
/*
 * FILL MATRIX BY USER INPUT
 */
void fillM(int m, int p, float **T) {
  for (int i = 0; i < m; i++) {
    for (int j = 0; j < p; j++) {
      T[i][j] = 0;
      printf("Enter coefficient for %p[%d][%d]", T, i, j);
      scanf("%f", &T[i][j]);
    }
  }
}
/*
 * FREE MATRIX
 */
void freeAll(float **T, int m) {
  for (int i = 0; i < m; i++) {
    free(T[i]);
  }
  free(T);
}
/*
 * IMPLEMENTATION OF '.' OPERATOR FOR MATRIX
 */
float **multiplication(float **M1, float **M2, int m, int q) {
  float **R = allocate(m, q);
  for (int i = 0; i < m; i++) {
    for (int j = 0; j < q; j++) {
      for (int k = 0; k < q; k++) {
        R[i][j] += M1[i][k] * M2[k][j];
      }
    }
  }
  return R;
}
/*
 * BUILD AUGMENTED MATRIX
 */
float **AugmentedMatrix(float **M1, float **M2, int m, int n) {
  float **A = allocate(m, m + 1);
  for (int i = 0; i < m; i++) {
    for (int j = 0; j < n + 1; j++) {
      (j != n) ? (A[i][j] = M1[i][j]) : (A[i][j] = M2[i][0]);
    }
  }
  return A;
}
/*
 * PERFORM GAUSS ALGORITHM ONLY ON AUGMENTED MATRIX
 */
void gauss(float **A, int m, int p) {
  if (m != p) {
    puts("La matrice doit etre carree !");
    return;
  }
  for (int k = 0; k <= m - 1; k++) {
    for (int i = k + 1; i < m; i++) {
      float pivot = A[i][k] / A[k][k];
      for (int j = k; j <= m; j++) {
        A[i][j] = A[i][j] - pivot * A[k][j];
      }
    }
  }
}
/*
 * DETERMINE ALL UNKNOWNS VARIABLES
 */
float *findSolutions(float **A, int m) {
  float *S = calloc(m, sizeof *S);
  S[m - 1] = A[m - 1][m] / A[m - 1][m - 1];
  for (int i = m - 1; i >= 0; i--) {
    S[i] = A[i][m];
    for (int j = i + 1; j < m; j++) {
      S[i] -= A[i][j] * S[j];
    }
    S[i] = S[i] / A[i][i];
  }
  return S;
}
int main() {
  int m, n, p, q;
  float **P, **Q, **B, **A, *S;
  clock_t start, end;
  double execution;
  puts("Nombre de ligne suivit du nombre de colonne pour la matrice 1:");
  scanf("%d%d", &m, &p);
  puts("Nombre de ligne suivit du nombre de colonne pour la matrice 2:");
  scanf("%d%d", &n, &q);
  start = clock();
  P = allocate(m, p);
  Q = allocate(n, q);
  fillM(m, p, P);
  fillM(n, q, Q);
  printMatrix(P, m, p);
  printMatrix(Q, n, q);
  B = multiplication(P, Q, m, n);
  printMatrix(B, m, q);
  A = AugmentedMatrix(P, B, m, p);
  gauss(A, m, p);
  printMatrix(A, m, m + 1);
  S = findSolutions(A, m);
  puts("SOLUTIONS");
  for (int i = 0; i < m; i++)
    printf("x%d = %f\n", i, S[i]);
  freeAll(P, m);
  freeAll(Q, n);
  freeAll(B, m);
  freeAll(A, m);
  free(S);
  end = clock();
  execution = ((double)(end - start) / CLOCKS_PER_SEC);
  printf("RUNTIME: %f seconds", execution / 10);
  return 0;
}

\end{lstlisting}
\subsubsection{Commentaires du code}
Mon implémentation utilise strictement l'algorithme de Gauss rappelé précédemment avec seulement quelques changements d'indices puisque au lieu de travailler sur une matrice carrée et un vecteur colonne, mon programme utilise une matrice augmentée ayant $m$ lignes et $m+1$ colonnes, $m\in \mathbb{N}^*$. \\
\textbf{Détail des fonctions non conventionnelles:}\\
\textit{Comme mentioné précédemment, je ne détaillerai pas les fonction gauss() et findSolutions() puisque ces fonctions permettent strictement que d'une part d'implémenter l'algorithme de Gauss et d'autre part à "remonter" la matrice échelonnée afin de récupérer les valeurs des inconnus}. \\
-\textbf{float **AugmentedMatrix(float **M1, float **M2, int m, int n):} cette fonction permet de créer une matrice $A \in \mathcal{M}_{m,m+1}$ à partir de la concaténation de $M1 \in \mathcal{M}_{mm}$ et $M2 \in \mathcal{M}_{m,1}$. \\
Soient $a_{ij}$ les coefficients peuplant $A$, $b_{ij}$ les coefficients peuplant $M1$ et $c_{i0}$ les coefficients peuplant $M2$. \\
On obtient alors $a_{ij} = b_{ij} \forall i \in \mathbb{N}_{m}, \forall j \in \mathbb{N}_{m}$ et $a_{ij} = c_{i0}$ si $j = m+1$. \\
Cette fonction renvoie alors $A$, la matrice de floattant créee dynamiquement.
\subsubsection{Inputs / Outputs}
Mon programme demande d'abord 4 entiers $m,p,n,q$ entiers qui correspondent aux dimensions de la première matrice $A \in \mathcal{M}_{mp}$ et de la seconde matrice $X \in \mathcal{M}_{nq}$. Le but étant de résoudre le système $AX=b$, nous initialiserons $X$ à $1$. Ce choix de valeur permettra de controller la validité du programme, ainsi si à la fin du programme $\forall x_i \neq 1, \forall i \in \mathbb{N}_{n}$, on pourra affirmer que le programme est faux.  \\
Sur le $m\times p$ prochaines lignes, le programme demande les coefficients de $A$. \\
Sur les $n \times q$ prochaines lignes, le programme demandera les coefficient du vecteur colonne $X$, que l'utilisateur initialisera à $1$. \\
On peut alors automatiser les entrée en utilisant des fichiers. \\
Ainsi la matrice:
$ M = \begin{pmatrix}
3 & 0 & 4\\
7 & 4 & 2 \\
-1 & 1 & 2
\end{pmatrix}
$
et 
$ X = \begin{pmatrix}
1 \\
1 \\
1 \\
\end{pmatrix}
$ \\
sont représentés par ce fichier d'entrée:
\newpage
\begin{lstlisting}[caption=input.txt]
3 3
3 1 
3 0 4
7 4 2
-1 1 2
1 1 1
\end{lstlisting}
Pour ce qui est des résultats produits par mon programme, une fois injecté dans un fichier texte, une input "type" ressemble à ceci. \\
\begin{lstlisting}[caption=Gauss elimination with M and X matrix]
PRINTING MATRIX FROM: 0x556d938672c0 LOCATION :
3.000000 0.000000 4.000000
7.000000 4.000000 2.000000
-1.000000 1.000000 2.000000
PRINTING MATRIX FROM: 0x556d93867340 LOCATION :
1.000000
1.000000
1.000000
PRINTING MATRIX FROM: 0x556d938673c0 LOCATION :
7.000000
13.000000
2.000000
PRINTING MATRIX FROM: 0x556d93867440 LOCATION :
3.000000 0.000000 4.000000 7.000000
0.000000 4.000000 -7.333333 -3.333332
0.000000 0.000000 5.166667 5.166667
SOLUTIONS
x0 = 1.000000
x1 = 1.000000
x2 = 1.000000
RUNTIME: 0.000002 seconds 
\end{lstlisting}
On remarquera que le programme affiche dans cet ordre: \\
- \textbf{La Matrice A} \\
- \textbf{La Matrice X} \\
- \textbf{La Matrice B trouvé avec les valeur de X} \\
- \textbf{La Matrice augmentée en triangle supérieur} \\
- \textbf{Les solutions} \\
- \textbf{Un timer permettant de contrôler le temps d'exécution approximatif de mon programme} 
\subsection{Exemples d'exécutions}

Soient $A_2 = \begin{pmatrix}
-3 & 3 & -6 \\
-4 & 7 &  8 \\
5 & 7 & -9 \\
\end{pmatrix}
$
,
$A_4 = \begin{pmatrix}
7 & 6 & 9 \\
4 & 5 &  -4\\
-7 & -3 & 8 \\
\end{pmatrix}
$,
$A_6 = \begin{pmatrix}
-3 & 3 & -6 \\
-4 & 7 &  8 \\
5 & 7 & -9 \\
\end{pmatrix}
$
\\
On obtient respectivement ces résulats:
\\
\begin{lstlisting}[caption=Matrix 2 results]
PRINTING MATRIX FROM: 0x55f604fb32c0 LOCATION :
-3.000000 3.000000 -6.000000
-4.000000 7.000000 8.000000
5.000000 7.000000 -9.000000
PRINTING MATRIX FROM: 0x55f604fb3340 LOCATION :
1.000000
1.000000
1.000000
PRINTING MATRIX FROM: 0x55f604fb33c0 LOCATION :
-6.000000
11.000000
3.000000
PRINTING MATRIX FROM: 0x55f604fb3440 LOCATION :
-3.000000 3.000000 -6.000000 -6.000000
0.000000 3.000000 16.000000 19.000000
0.000000 0.000000 -83.000000 -83.000000
SOLUTIONS
x0 = 1.000000
x1 = 1.000000
x2 = 1.000000
RUNTIME: 0.000002 seconds   
\end{lstlisting}
\begin{lstlisting}[caption=Matrix 4 results]
PRINTING MATRIX FROM: 0x55f7afd662c0 LOCATION :
7.000000 6.000000 9.000000
4.000000 5.000000 -4.000000
-7.000000 -3.000000 8.000000
PRINTING MATRIX FROM: 0x55f7afd66340 LOCATION :
1.000000
1.000000
1.000000
PRINTING MATRIX FROM: 0x55f7afd663c0 LOCATION :
22.000000
5.000000
-2.000000
PRINTING MATRIX FROM: 0x55f7afd66440 LOCATION :
7.000000 6.000000 9.000000 22.000000
0.000000 1.571428 -9.142858 -7.571429
0.000000 0.000000 34.454552 34.454548
SOLUTIONS
x0 = 1.000001
x1 = 0.999999
x2 = 1.000000
RUNTIME: 0.000002 seconds  
\end{lstlisting}
\begin{lstlisting}[caption=Matrix 6 results]
PRINTING MATRIX FROM: 0x557fdaa552c0 LOCATION :
3.000000 -1.000000 0.000000
0.000000 3.000000 -1.000000
0.000000 -2.000000 3.000000
PRINTING MATRIX FROM: 0x557fdaa55340 LOCATION :
1.000000
1.000000
1.000000
PRINTING MATRIX FROM: 0x557fdaa553c0 LOCATION :
2.000000
2.000000
1.000000
PRINTING MATRIX FROM: 0x557fdaa55440 LOCATION :
3.000000 -1.000000 0.000000 2.000000
0.000000 3.000000 -1.000000 2.000000
0.000000 0.000000 2.333333 2.333333
SOLUTIONS
x0 = 1.000000
x1 = 1.000000
x2 = 1.000000
RUNTIME: 0.000002 seconds                       
\end{lstlisting}
\textbf{On remarquera} que sur le calcul de $A_4$, on tombe sur des valeur extrêmement proche de $1$. Ceci est provoqué à cause des erreurs d'arrondis provoqués par l'encodage des nombres flottants.
\end{document}      
\\
\textbf{Nous remarquerons} que l'implémentation utilisant la matrice augmentée est sensiblement meilleure en terme d'efficacité. En effet, le temps d'exécution est multiplié par $100$ sur la première implémentation.