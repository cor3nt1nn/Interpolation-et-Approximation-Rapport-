\documentclass{report}
\usepackage[T1]{fontenc}
\usepackage[utf8]{inputenc}
\usepackage{mathtools}
\usepackage{amssymb}
\usepackage{hyperref}
\usepackage{float}
\usepackage{amsthm}
\usepackage{listings}
\usepackage{geometry}
\usepackage{setspace}
\usepackage{graphicx}
\usepackage{fancyhdr}
\usepackage{subcaption}
\usepackage{cleveref}

\begin{document}
\section{Résolution de systèmes linéaires par des Méthodes Directes}
Dans le cadre de ce premier TP, nous devions implémenter l'algorithme du \emph{Pivot de Gauss} en utilisant le langage de programmation C.
\subsection{Détail de l'algorithme}
Soit $A$ une matrice $ \in \mathcal{M}_{m,m}\,m \in \mathbb{N}$ et $b \in \mathcal{M}_{m,1}$ . \\
L'algorithme de Gauss se décrit ainsi: \\ \\
\begin{lstlisting}[mathescape=true]
Pour $k = 1,\ldots,n-1$ Faire:
	Pour $i=k+1,\ldots,n$ Faire:

				$\alpha_i^{(k)} = \frac{a_{ik}^{(k)}}{a_{kk}^{(k)}}$ 
		
		Pour $j=k,\ldots,n$ Faire:

		   		$a_{ij}^{(k+1)} = a_{ij}^{(k)} - \alpha_i^{(k)}a_{kj}^{(k)}$
	
		FIN Pour $j$

				$b_i^{(k+1)} = b_i^{(k)} - \alpha_i^{(k)}b_k^{(k)}$

	FIN Pour $i$
FIN Pour $k$
\end{lstlisting}
Après cette algorithme permettant l'échelonnage de la matrice, pour trouver les solutions du système, on appliquera la formule suivante: \\
\begin{Large}
$ x_n = \frac{b_n}{a_{m,m}}$ \\
\end{Large}
\Large{et} \\ 
\begin{Large}
$ \forall i = n-1, \ldots, 1$, $x_i = \frac{1}{a_{ii}}\left( b_i-\sum\limits_{j=1+i}^n a_{ij}x_j \right)$
\end{Large} \\ \\
\normalsize
La complexité temporelle de cet algorithme est cubique soit $O(n^3)$ avec une compléxité exacte de $\frac{2n^3}{3}$. \\
Pour l'implémentation de cet algorithme, nous allons présenter deux façons de le conceptualiser avec une comparaison algorithmique des deux programmes. \\
\subsection{Implémentation du pivot de Gauss sans matrice Augmentée}
% \import{le fichier}
\subsection{Implementation du pivot de Gauss avec matrice Augmentée}
\end{document}