\documentclass{report}
\setlength{\headheight}{24.1638pt}
%packages
\usepackage[french]{babel}
\usepackage[T1]{fontenc}
\usepackage[utf8]{inputenc}
\usepackage{mathtools}
\usepackage{amssymb}
\usepackage{hyperref}
\usepackage{float}
\usepackage{amsthm}
\usepackage{listings}
\usepackage{geometry}
\usepackage{setspace}
\usepackage{graphicx}
\usepackage{fancyhdr}
\usepackage{subcaption}
\usepackage{cleveref}

%commands
\newtheorem{defi}{Définition}
\renewcommand{\thedefi}{\empty{}}

\renewcommand\headrulewidth{1pt}
\newcommand{\crule}[3][c]{%
    \par\noindent
    \makebox[\linewidth][#1]{\rule{#2\linewidth}{#3}}}
\renewcommand{\thechapter}{\Roman{chapter}}

%Style de page
\pagestyle{fancy}
\fancyhead[L]{}
\fancyhead[C]{}
\fancyhead[R]{\leftmark}
\allowdisplaybreaks
\geometry{hmargin=3cm,vmargin=2.5cm}


%préambule
\begin{document}
\section{Méthode de Jacobi}
Rappelons que la méthode de \textbf{Jacobi} est itérative et ne garantit pas toujours un résultat. La méthode est définie si A est définie positive.\\
L'algorithme permet de trouver un résultat si la matrice est dites à diagonale strictement dominante. \\
Autrement dit, Soit $[a_{ij}]_{0 \leq i,j \leq n}$ les coefficients réels peuplant $A \in \mathcal{M}_{n,n} (\mathbb{R})$, alors si:\\
$\forall i, \vert a_{i,i} \vert <  \sum \limits_{i \neq j} \vert a_{ij} \vert$, on a que Jacobi converge vers l'unique solution du système $Ax=b$. 
\subsection{Principe de la méthode}
On veut résoudre $Ax=b$ avec $A \in \mathcal{M}_{n,n} (\mathbb{R})$, $n \in \mathbb{N}$, $x$ la vecteur colonne contenant les inconnus et $b$ le vecteur colonne des solution. \\
On pose $D \in \mathcal{M}_{n,n}(\mathbb{R})$ la matrice contenant les coefficients $[a_{i,j}]_{0 \leq i=j \leq n}$ de $A$. \\
On pose aussi $E$ et $F$ avec $E$ la matrice triangulaire opposée inférieure de $A$ et $F$ la matrice supérieure opposée de $A$.  \\
On obtient alors:
\begin{align}
Ax &=b \\
(D-E-F)x &= b \\
Dx -(E+F)x &= b \\
x &=D^{-1}(E+F)x+D^{-1}b \\
x^{k+1} &=D^{-1}(E+F)x^{k}+D^{-1}b 
\end{align}
Ce qui donne l'algorithme suivant: \\
Soit $\epsilon$ L'erreur maximale, un point initial $x^0$ et $k=0$ \\
avec $\epsilon^{0} = \vert \vert A x^0 -b \vert \vert$ \\
On obtient: 
\begin{lstlisting}[mathescape=true, frame=single]
Tant que ($\epsilon^{(k)} \leq \epsilon$)
$x^{k+1}_i = \frac{1}{a_{ii}} [ b_i - \sum \limits_{j\neq i} a_{ij}x_j^{(k)}], i=1,\ldots,n$
$\epsilon^{k+1} = \vert \vert Ax^{k+1} -b \vert \vert$
$k=k+1$
FIN JACOBI
\end{lstlisting}
\textbf{Remarque, on ajoutera aussi un nombre d'iterations maximum afin de ne pas être dans le cas d'une boucle infinies (si jacobi diverge alors l'erreur augmente)}.
\subsection{Résolution manuelle}
\textit{Nous en détaillerons seulement une itération} \\
Soit
$A= \begin{pmatrix}
4 & 1 & 0 \\
-1 & 3 & 6 \\
-2 & -5 & -3
\end{pmatrix}$,
$b = \begin{pmatrix}
8 \\
3 \\
8 \\
\end{pmatrix}$
,$x= \begin{pmatrix}
x_1 \\
x_2 \\
x_3
\end{pmatrix}$ 
et $x^{(0)} = \begin{pmatrix}
0 \\
0\\
0\\
\end{pmatrix}$ \\
On a $A = \underbrace{\begin{pmatrix}
4 & 0 & 0 \\
0 & 3 & 0 \\
0 & 0 & -2
\end{pmatrix}}_{D}
- 
\underbrace{\begin{pmatrix}
0 & 0& 0 \\
1 & 0 & 0 \\
2 & 5 &0 
\end{pmatrix}}_E
- \underbrace{\begin{pmatrix}
0 & -1 & 0 \\
0 & 0 & -6\\
0 & 0 & 0
\end{pmatrix}}_F$ \\
On a donc $x^{k+1}= D^{-1} [(E+F)x^{k} + b]$ \\
Dans le cas présent on obtient alors: \\
$x^1 = \begin{pmatrix}
\frac{1}{4} & 0 & 0 \\
0 & \frac{1}{3} & 0 \\
0 & 0 & -\frac{1}{2}
\end{pmatrix}
\left[ \left[ \begin{pmatrix}
0 & 0 & 0 \\
1 & 0  & 0 \\
2 & 5 & 0 \\
\end{pmatrix} + 
\begin{pmatrix}
0 & -1 & 0 \\
0 & 0 & -6 \\
0 & 0 & 0 
\end{pmatrix}
\right]
\begin{pmatrix}
0\\
0\\
0
\end{pmatrix}
+
\begin{pmatrix}
8 \\
3\\
8
\end{pmatrix}
  \right]$ \\
$x^1 = \begin{pmatrix}
2 \\
1 \\
-4
\end{pmatrix}$ \\
et
\begin{align*}
\epsilon^{(1)} & = \vert \vert Ax^{(1)} -b \vert \vert \\
\epsilon^{(1)} & = \vert \vert \begin{pmatrix}
4 & 1 & 0 \\
-1 & 3 & 6 \\
-2 & -5 & -3
\end{pmatrix}
\begin{pmatrix}
2\\
1\\
-4
\end{pmatrix} - 
\begin{pmatrix}
8 \\
3 \\
8
\end{pmatrix} \vert \vert \\
\epsilon^{(1)} & = \vert \vert \begin{pmatrix}
1 \\
-26 \\
-5
\end{pmatrix} \vert \vert \\
\epsilon^{(1)} & =  \sqrt{1^2+(-26)^2+(-5)^2} \\
\epsilon^{(1)} & =  \sqrt{(702)} \\
\end{align*}

\subsection{Implémentation}
Pour l'implémentation de cette méthode, nous utiliserons $\epsilon$ comme suit: \\
$\epsilon^{(k)} = p^k = \text{Max}_{i=1,\ldots, n} \vert \bar{x_i} -\tilde{x_i}^k \vert $ \\
Où $\bar{x_i}$ est les résultat attendu et $\tilde{x_i}^k$ est l'approximation trouvée à l'étape $k$. \\
De plus on utilisera aussi une limite d'occurrence, pour pouvoir gérer les matrices où \textbf{Jacobi} diverge.
\end{document}                                                                                                                                                                                                                                              