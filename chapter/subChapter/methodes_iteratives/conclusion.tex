% \documentclass{report}
% \setlength{\headheight}{24.1638pt}
% %packages
% \usepackage[french]{babel}
% \usepackage[T1]{fontenc}
% \usepackage[utf8]{inputenc}
% \usepackage{mathtools}
% \usepackage{amssymb}
% \usepackage{hyperref}
% \usepackage{float}
% \usepackage{amsthm}
% \usepackage{listings}
% \usepackage{geometry}
% \usepackage{setspace}
% \usepackage{graphicx}
% \usepackage{fancyhdr}
% \usepackage{subcaption}
% \usepackage{cleveref}

% %commands
% \newtheorem{defi}{Définition}
% \renewcommand{\thedefi}{\empty{}}

% \renewcommand\headrulewidth{1pt}
% \newcommand{\crule}[3][c]{%
%     \par\noindent
%     \makebox[\linewidth][#1]{\rule{#2\linewidth}{#3}}}
% \renewcommand{\thechapter}{\Roman{chapter}}

% %Style de page
% \pagestyle{fancy}
% \fancyhead[L]{}
% \fancyhead[C]{}
% \fancyhead[R]{\leftmark}
% \allowdisplaybreaks
% \geometry{hmargin=3cm,vmargin=2.5cm}

% %préambule
% \begin{document}
\section{Conclusion Générale des Méthodes Itératives}
\subsection{Tableau récapitulatif}
$
\begin{array}{|c|c|c|c|c|}
\hline
A_i & p(J) & \text{NbIterJacobi} & p(GS) & \text{NbIterGauss-Seidel} \\
\hline
A_1 & \infty & 742 & \infty & 192\\
\hline
A_2 & \infty & 252 & \infty & 128 \\
\hline
A_3 & 0.000001& 19 & 0.00000 & 9 \\
\hline
A_4 & 0.000001 & 35 & 0.000001& 59 \\
\hline
A_5 & 0.000001 & 25 & 0.000001& 14 \\
\hline
A_6 & 0.000002 & 26 & 0.000001 & 15 \\
\hline
A_7 & 0.000001 & 26 & 0.000001 & 15 \\
\hline
A_8 & 0.000002 & 26 & 0.000001 & 15 \\
\hline
A_9 & 0.000002 & 35 & 0.000001 & 19 \\ 
\hline
A_{10} & 0.000002 & 88 & 0.000001 & 44 \\
\hline
A_{11} & 0.000001 & 120 & 0.000001& 60 \\
\hline
A_{12} & 0.000001 & 140 & 0.000001 & 73 \\
\hline
\end{array}
$
\subsection{Conclusion}
Comme peuvent le démontrer les différents graphiques ainsi que le tableau ci-dessus, nous remarquerons que la Méthode de \textbf{Gauss-Seidel} reste majoritairement plus efficace que la méthode de \textbf{Jacobi}. \\
Nous insisterons sur le fait que la Méthode de \textbf{Gauss-Seidel} est particulièrement adaptée pour le calcul parallèle alors que la méthode de \textbf{Jacobi} est plus adaptée sur des matrices creuses.
% \end{document}