\documentclass{report}
\setlength{\headheight}{24.1638pt}
%packages
\usepackage[french]{babel}
\usepackage[T1]{fontenc}
\usepackage[utf8]{inputenc}
\usepackage{mathtools}
\usepackage{amssymb}
\usepackage{hyperref}
\usepackage{float}
\usepackage{amsthm}
\usepackage{listings}
\usepackage{geometry}
\usepackage{setspace}
\usepackage{graphicx}
\usepackage{fancyhdr}
\usepackage{subcaption}
\usepackage{cleveref}

%commands
\newtheorem{defi}{Définition}
\renewcommand{\thedefi}{\empty{}}

\renewcommand\headrulewidth{1pt}
\newcommand{\crule}[3][c]{%
    \par\noindent
    \makebox[\linewidth][#1]{\rule{#2\linewidth}{#3}}}
\renewcommand{\thechapter}{\Roman{chapter}}

%Style de page
\pagestyle{fancy}
\fancyhead[L]{}
\fancyhead[C]{}
\fancyhead[R]{\leftmark}
\allowdisplaybreaks
\geometry{hmargin=3cm,vmargin=2.5cm}


%préambule
\begin{document}

\lstset{
  firstnumber=0, 
  numbers=left,               
  frame=single,
  language=C,                                       
  showstringspaces=false
}
\section{Méthode de Gauss-Seidel}
\subsection{Introduction à la méthode de Gauss-Seidel}
La méthode de Gauss-Seidel est une méthode itérative pour résoudre 
les systèmes linéaires de la forme $Ax=b$, où $A$ est une matrice carrée d'ordre $n$ et $x, b$ sont des vecteurs de $\mathbb{R}^n$. 
C'est une méthode qui génère 
une suite qui converge vers la solution de ce système lorsque celle-ci en a une et lorsque les conditions de convergence suivantes sont satisfaites (quels que soient le vecteur $b$ et le point initial $x^0$):
\begin{itemize}
  \item Si la matrice $A$ est symétrique définie positive,
  \item Si la matrice $A$ est à diagonale strictement dominante.
\end{itemize}
\subsection{Mise en place des matrices pour la méthode de Gauss-Seidel}
Soit $Ax=b$ le système linéaire à résoudre, où $A\in \mathcal{M}_{n,n}$ et $b\in  \mathcal{M}_{n,1}$. On cherche $x\in \mathcal{M}_{n,1}$ solution du système.
Dans un premier temps, on va écrire $A$ sous la forme $A=D-E-F$ où $D$ est une matrice diagonale, $E$ est une matrice triangulaire inférieure, et $F$ est une matrice triangulaire supérieure. \\
On peut alors écrire:
\begin{align}
  Ax&=b \\
  \Leftrightarrow  (D-E-F)x&=b \\
  \Leftrightarrow  Dx&=b-(E+F)x \\
  \Leftrightarrow  x&=D^{-1}[b-(E+F)x]
\end{align}
On définit ensuite une suite de vecteurs $(x^k)$ en choisissant un vecteur $x^0$ et par la formule de récurrence:\\
\begin{equation}
  x_i^{k+1}=\frac{1}{a_{i,i}}\Bigg(b_i-\sum \limits_{j = 1}^{i-1}a_{i,j}x_{j}^{k+1} - \sum \limits_{j = i+1}^{n}a_{i,j}x_{j}^{k}\Bigg)
\end{equation}
\subsection{Algorithme}
Pour résoudre un système $Ax=b$, avec $A \in \mathcal{M}_{n}$ et $b\in \mathcal{M}_{n,1}$, on s'appuie sur l'algorithme suivant en posant :
\begin{itemize}
  \item un vecteur initial $x^{(0)}$ choisi au préalable,
  \item l'erreur à l'itération k=0 calculée par $\varepsilon^{(0)}=\Vert Ax^{(0)}-b\Vert$
\end{itemize}\vspace{6pt}
\begin{lstlisting}[mathescape=true, frame=single, basicstyle=\linespread{1.5}\fontsize{8}{10}\selectfont]
$x^{(0)}=x_0 \in \mathcal{M}_{n,1}$
$\varepsilon^{(0)}=\varepsilon \text{ (erreur)}$
$k=0$
Tant Que $(\varepsilon^{(k)} >= \varepsilon)$ faire:
      Pour $i=1$ a $n$:
            $x_{i}^{k+1}=\frac{1}{a_{i,i}}\Bigg[b_i-\Bigg(\sum \limits_{j=i+1}^{n}a_{i,j}x_j^{(k)} + \sum \limits_{j=1}^{i-1}a_{i,j}x_j^{(k+1)}\Bigg)\Bigg]   \text{  pour  } i=1, ..., n$
      $\varepsilon^{(k+1)}=\Vert Ax^{(k+1)}-b\Vert$
      $k=k+1$
Fin Tant Que
\end{lstlisting}
\subsection{Résolution manuelle}
\subsection{Implémentation}
\subsection{Exemples d'exécution}
\end{document}

