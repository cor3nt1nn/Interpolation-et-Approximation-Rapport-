\lstset{
  firstnumber=0, 
  numbers=left,               
  frame=single,
  language=C,                                       
  showstringspaces=false
}
\section{Méthode de Gauss-Seidel}
\subsection{Introduction à la méthode de Gauss-Seidel}
La méthode de Gauss-Seidel est une méthode itérative pour résoudre 
les systèmes linéaires de la forme $Ax=b$, où $A$ est une matrice carrée d'ordre $n$ et $x, b$ sont des vecteurs de $\mathbb{R}^n$. 
C'est une méthode qui génère 
une suite qui converge vers la solution de ce système lorsque celle-ci en a une et lorsque les conditions de convergence suivantes sont satisfaites (quels que soient le vecteur $b$ et le point initial $x^0$):
\begin{itemize}
  \item Si la matrice $A$ est symétrique définie positive,
  \item Si la matrice $A$ est à diagonale strictement dominante.
\end{itemize}
\subsection{Mise en place des matrices pour la méthode de Gauss-Seidel}
Soit $Ax=b$ le système linéaire à résoudre, où $A\in \mathcal{M}_{n,n}$ et $b\in  \mathcal{M}_{n,1}$. On cherche $x\in \mathcal{M}_{n,1}$ solution du système.
Dans un premier temps, on va écrire $A$ sous la forme $A=D-E-F$ où $D$ est une matrice diagonale, $E$ est une matrice triangulaire inférieure, et $F$ est une matrice triangulaire supérieure. \\
On peut alors écrire:
\begin{align}
  Ax&=b \\
  \Leftrightarrow  (D-E-F)x&=b \\
  \Leftrightarrow  Dx&=b-(E+F)x \\
  \Leftrightarrow  x&=D^{-1}[b-(E+F)x]
\end{align}
On définit ensuite une suite de vecteurs $(x^k)$ en choisissant un vecteur $x^0$ et par la formule de récurrence:\\
\begin{equation}
  x_i^{k+1}=\frac{1}{a_{i,i}}\Bigg(b_i-\sum \limits_{j = 1}^{i-1}a_{i,j}x_{j}^{k+1} - \sum \limits_{j = i+1}^{n}a_{i,j}x_{j}^{k}\Bigg)
\end{equation}
\subsection{Algorithme}
\subsection{Résolution manuelle}

