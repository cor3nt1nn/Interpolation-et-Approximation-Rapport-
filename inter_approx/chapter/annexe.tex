\chapter*{Annexe}
\addcontentsline{toc}{chapter}{Annexe}
\label{annexe}
\section*{Jeux d'essais}
\addcontentsline{toc}{section}{Jeux d'essais}
\subsection*{Densité ($D$) de l'eau en fonction de la température ($T$)}
\centering
\begin{tabular}{|c|c|c|c|c|c|c|c|c|c|c|}
    \hline
    T( C) & 0 & 2 & 4 & 6 & 8 & 10 & 12 & 14 & 16 & 18 \\
    \hline
    D($t/m^3$) &0.99897&0.99846& 0.99987&0.99997&1.00000&0.99997&0.99988&0.99973&0.99953&0.99927\\
    \hline
\end{tabular}
\raggedright
\begin{tabular}{|c|c|c|c|c|c|c|c|c|c|c|}
    \hline
    T( C) & 20 & 22 & 24 & 26 & 28 & 30 & 32 & 34 & 36 & 38 \\
    \hline
    D($t/m^3$) &0.99805&0.999751&0.99705&0.99650&0.99664&0.99533&0.99472&0.99472&0.99333&0.99326 \\
    \hline
\end{tabular}

\subsection*{Dépenses mensuelles et revenus}
On s'intéresse à la relation qui existe entre les Dépenses de loisirs mensuelles D et les revenus R des employés d'une entreprise.\vspace{8pt}\\

\begin{tabular}{|c|c|c|c|c|c|c|c|c|c|c|c|}
    \hline
    R & 752 & 855 & 871 & 734 & 610 & 582 & 921 & 492 & 569 & 462 & 907\\
    \hline
    D &85&83&162&79&81&83&281&81&81&80&243\\
    \hline
\end{tabular}
\begin{tabular}{|c|c|c|c|c|c|c|c|c|c|c|c|}
    \hline
    R & 643&862&524&679&902&918&828&875&809&894&\\
    \hline
    D &84&84&82&80&226&260&82&186&77&223&\\
    \hline
\end{tabular}
\subsection*{Série $S$ dûe à Anscombe}

\begin{tabular}{|c|c|c|c|c|c|c|c|c|c|c|c|}
    \hline
    $x_i$ & 10 &  8  & 13&  9  & 11  &14  &6  & 4 &  12  &7  & 5 \\
    \hline
    $y_i$ &8.04&6.95&7.58&8.81&8.33&9.96&7.24&4.26&10.84&4.82&5.68\\
    \hline
\end{tabular}
\subsection*{Série chronologique avec accroissement exponentiel}
\begin{tabular}{|c|c|c|c|c|c|c|c|c|c|c|c|}
    \hline
    $x_i$ &88&89&90&91&92&93&94&95&96&97 \\
    \hline
    $y_i$ &5.89&6.77&7.87&9.11&10.56&12.27&13.92&15.72&17.91&22.13\\
    \hline
\end{tabular}
\subsection*{Vérification de la loi de Pareto}
Loi de Pareto: \textit{"Entre le revenu x et le nombre y de personnes ayant un revenu supérieur à x, il existe une relation du type:}\\
\begin{center}
    $y=\frac{A}{x^a}=Ax^{-a}$
\end{center}
où $a$ et $A$ sont des constantes positives caractéristiques de la région considérée et de la période étudiée. \vspace{8pt}\\
\begin{tabular}{|c|c|c|c|c|c|c|c|}
    \hline
    $x_i$ &20&30&40&50&100&300&500 \\
    \hline
    $y_i$ &352&128&62.3&35.7&6.3&0.4&0.1\\
    \hline
\end{tabular}

