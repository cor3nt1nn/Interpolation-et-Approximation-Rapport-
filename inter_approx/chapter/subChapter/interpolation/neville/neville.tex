\section{Interpolation par la méthode de Neville}
\subsection{Présentation de la méthode}
La méthode de Neville est une technique d'interpolation qui permet d'approximer une fonction inconnue à partir de données discrètes. Elle repose sur un processus récursif de construction d'un polynôme interpolateur à partir des données initiales. À chaque étape, deux polynômes voisins sont combinés pour former un nouveau polynôme qui passe par certains points données. Cette méthode devient rapidement imprécise au fur et à mesure que le nombre de points augmente. Elle est en revanche efficace pour l'interpolation de petits ensembles de données.\vspace{6pt}\\
Considérons un ensemble de $n$ points donnés, notés $(x_i, y_i)$, où les $x_i$ sont deux à deux distincts. Nous cherchons à déterminer un polynôme d'interpolation $p(x)$ de degré $n-1$ au maximum, qui satisfait la condition suivante :
\begin{center}
    $p(x_i)=y_i$, \text{ avec   } $i=0, ..., n-1$
\end{center}
La méthode de Neville consiste à évaluer ce polynôme pour le point d'abscisse $x$.\\
Soit $p_k[x_i, ..., x_i+k](x)$ le polynôme de degré $k$ qui passe par les points $(x_i, y_i), ..., (x_{i+k}, y_{i+k})$. Alors $p_k[x_i, ..., x_{i+k}](x)$ vérifie la relation de récurrence suivante:\vspace{5pt}\\
\begin{equation*}
    \begin{cases}
        p_0[x_i](x)=y_i, \text{     avec  } 0\leqslant i< n \text{  et } k=0\vspace{6pt}\\
        p_k[x_i, ..., x_{i+k}](x)=\frac{(x-x_{i+k})p_{k-1}[x_i, ..., x_{i+k-1}](x)+(x_i-x)p_{k-1}[x_{i+1}, ..., x_{i+k}](x)}{x_i-x_{i+k}}, \text{     avec  } 1\leqslant k<n \text{ et } 0\leqslant i< n
    \end{cases}
\end{equation*}
Cette relation de récurrence permet de calculer $p_{n-1}[x_0, ..., x_{n-1}](x)$, qui est le polynôme recherché.
% Arbre à faire
\subsection{Résolution Manuelle}
\subsection{Algorithme}
\subsection{Implémentation en C}
\subsection{Exemples d'exécution}